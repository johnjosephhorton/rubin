%% LyX 2.4.2.1 created this file.  For more info, see https://www.lyx.org/.
%% Do not edit unless you really know what you are doing.
\documentclass[english]{article}
\usepackage[T1]{fontenc}
\usepackage[latin9]{inputenc}
\usepackage{float}
\usepackage{amsmath}
\usepackage[a4paper]{geometry}
\geometry{verbose,tmargin=2cm,bmargin=2cm,lmargin=2cm,rmargin=2cm,footskip=0.75cm}
\PassOptionsToPackage{normalem}{ulem}
\usepackage{ulem}
\usepackage{babel}
\begin{document}
\setcounter{page}{0}
\title{General Equilibrium Sketch (CES)}

\maketitle

\pagenumbering{arabic}

I provide a sketch of the version we had in the previous draft with
some modifications. At the end, I give one or two ideas about other
ways we can think about modifying and extending the model. We can
discuss further on Monday.

\section{High-Level Discussion}

There are two ways one can think of writing a production function
in our setup. Below I introduce both options and discuss the pros
and cons of each. For this discussion suppose each occupation is comprised
by a set of tasks $\mathcal{H}$ that are done manually and a set
of AI chained tasks $\mathcal{A}$ done by the help of AI. At this
point we don't care how tasks have been allocated to human or machine.
\begin{enumerate}
\item Leontief production function of manual as well as AI-performed tasks
(a la Peyman model). This version treats each task as a separate input
for production. Suppose tasks indexed $1,\cdots,m$ are done by human
(i.e., they belong to set $\mathcal{H}$) and tasks indexed $m+1,\cdots,n$
are done via AI (i.e., belong to the set $\mathcal{A}$). We have:
\[
X=\min(t_{1}^{H},\cdots,t_{m}^{H},t_{m+1}^{AI},\cdots,t_{n}^{AI}).
\]

\begin{itemize}
\item Pros: 
\begin{itemize}
\item Implicitly assumes task durations are fixed (i.e., takes task parameters
as given). This is necessary assumption for the DP algorithm to solve
the job design problem. To produce one unit of good $X$ each task
$1,\cdots,n$ (whether manual or AI-performed) must be done exactly
once. This is closer to the current notion of task-based production
we currently have in the draft.
\item Accommodates the hand-off costs friction as a source of determining
job boundaries (as hand-offs are assumed to be a fixed time duration
property of tasks).
\end{itemize}
\item Cons: 
\begin{itemize}
\item Production (and GE) becomes too stylistic. Production function really
has only one input: labor. 
\item The equilibrium allocations and prices are solely determined by intrinsic
task parameters, not relative prices of inputs.
\item Use of AI is not endogenized based on equilibrium prices. Rather it's
determined based on intrinsic task parameter, and perhaps outside
of the model (say, for example, through some investment decision in
the ``first stage'' of a two-stage game where the level of $\alpha$
is decided in the first stage and the job design problem is solved
+ production takes place in the second stage). 
\item The GE doesn't have meaningful economics given how it's structured.
Hard to sell as a real production model as it doesn't fit to into
the literature.
\end{itemize}
\end{itemize}
\item Have a CES production function with AI and human as separate inputs
(a la Mert model).  Every AI chain and manual task enters the CES
aggregator separately: 
\[
X=\left(\sum_{a\in\mathcal{A}}\eta_{a}\left(t_{a}^{AI}\alpha^{-d_{a}}\right)^{\rho}+\sum_{h\in\mathcal{H}}\eta_{h}\left(t_{h}^{H}\right)^{\rho}\right)^{\frac{1}{\rho}}.
\]
Notice that if we fix the difficulty $d_{a}$ across all AI chains,
the this formulation implicitly assumes a Leontief production between
the AI input $\alpha$ and the human labor input $t_{a}^{AI}$ for
every AI chain.
\begin{itemize}
\item Pros: 
\begin{itemize}
\item Similar to conventional production functions. 
\item Three components 1) human labor for manual tasks, 2) human labor for
managed AI chains, 3) AI itself nested into two separable inputs:
1) human labor, 2) AI labor. 
\end{itemize}
\item Cons: 
\begin{itemize}
\item Assumes task durations are flexible. This is at odds with the assumptions
we are using for the job design problem where we treat task durations
as fixed. With this production function in one scenario we can employ
2 units of task A and 0.5 unit of task B whereas in another execution
plan we can employ 1 unit of each task. This is inconsistent with
the cost minimization characterization we had earlier.
\item This will make integration of firm's cost minimization problem and
GE model hard (if not impossible). One idea is to look at firm's profit
maximization as a two stage problem: one discrete and one continuous.
In the outer loop (discrete) we choose the allocation of tasks to
jobs. In the inner loop (continuous) we solve for the optimal level
of each input taking job assignments as given.
\end{itemize}
\end{itemize}
\end{enumerate}

\section{Task-based Production Environment}

We have a task-based model of production, with the following characteristics:
\begin{itemize}
\item Production of a consumption good $X$ requires completing an ordered
set of tasks, denoted by:
\[
\mathcal{T}=\{1,2,...,n\}.
\]
We refer to $\mathcal{T}$ as the ``production process'' of the
good. Later in the GE section where we have more than a single good
we use subscript $k$ for all variables introduced in this section,
but for now we are focusing on only a single good--so no subscripts.
\item Each task can be completed either manually or using an AI. If done
by AI, the task is a part of a chain of one or more tasks called an
``AI Chain.'' Denote the set of manual tasks done by human with
$\mathcal{H}$ and AI chains with $\mathcal{A}$. A task $i$ is either
in $\mathcal{H}$ or $\mathcal{A}$. We refer to members of these
sets as a ``module'': a manual tasks is one module; an entire AI
chain is also a module.
\item Each task $i\in\mathcal{T}$ has five cost parameters: human and AI
skill (i.e., human capital), human and AI time (i.e., labor), and
hand-off (or context-switching) cost. Denote the (human skill, machine
skill, human time, machine time, hand-off time) costs of task $i$
with $(c_{i}^{H},c_{i}^{AI},t_{i}^{H},t_{i}^{AI},t_{i}^{s})$. The
skill costs $c_{i}^{x}$ ($x\in\{H,AI\}$) are a one-off cost that
paid by the firm to train the worker to do task $i$. Once trained,
the firm employs the worker for the duration of task's time cost $t_{i}^{x}$
to produce of one unit of the good.\\
\\
Hand-off cost $t_{i}^{s}$ is a frictional cost that a worker must
pay to hand-off output of his/her task to the next worker. Hand-off
can be viewed as an additional task at the end of task $i$ that takes
time $t_{i}^{s}$ to complete but requires no additional skills. We
discuss hand-off costs a bit more after introducing jobs next.
\item Firms assign tasks to workers in ``jobs.'' A job is a contiguous
set of tasks in the production process $\mathcal{T}$. Jobs are mutually
exclusive and collectively cover all tasks in $\mathcal{T}$. Let
$\mathcal{J}_{j}=\{\mathcal{H}_{j},\mathcal{A}_{j}\}$ denote the
set of manual tasks and AI chains associated with job $j$ assigned
to worker $j$ (each worker is assigned to one job, so we can use
the same index for the worker and their job). Formally, assuming the
firm splits the production process into $J$ distinct jobs we have:
\begin{align*}
\forall j\neq j^{'},\quad j,j^{'}\in\{1,2,...,J\}: & \qquad\mathcal{J}_{j}\cap\mathcal{J}_{j^{'}}=\emptyset,\\
\end{align*}
and
\[
\bigcup_{j=1}^{J}\mathcal{J}_{j}=\mathcal{T}.
\]
\item Hand-off cost of a task is only realized when the task falls at the
boundary of two consecutive jobs $j$ and $j+1$. That is, if task
$i$ is the last task in job $j$, work $j$ incurs an additional
time cost $t_{i}^{s}$ to hand-off the output to the next worker.
Every task other than $i$ in job $j$ has zero realized hand-off
time cost. With a slight abuse of notation, we show hand-off cost
of job $j$ with $t_{j}^{s}:=t_{i}^{s}$. \\
\\
Remember that even though we are using notation $t_{j}^{s}$ to show
the hand-off time cost of job $j$ it is an inherent property of the
\uline{last task of the job}, not the job itself. So, $t_{j}^{s}$
is actually determined by the task at the boundary of job $j$, and
may differ from one job structure to another.
\item Continuing on the previous point, hand-off costs and thus production
costs depend on structure of jobs or ``job design.'' But how are
costs determined?\\
Consider some collection of jobs $\mathcal{J}=\{\mathcal{J}_{j}\}_{j=1}^{J}=\{\mathcal{H}_{j},\mathcal{A}_{j}\}_{j=1}^{J}$
containing $J$ distinct jobs. The wage that has to be paid to worker
employed in job $j$ is given by
\begin{equation}
w_{j}:=v_{L}+\sum_{a\in\mathcal{A}_{j}}c_{a}^{AI}+\sum_{h\in\mathcal{H}_{j}}c_{h}^{H},\label{eq:wage}
\end{equation}
where $v_{L}$ is worker's value of leisure.\footnote{To make workers indifferent between working at different jobs the
firm compensates them for the job's required skill costs. Moreover,
the firm pays workers the value of their leisure to make them indifferent
between working and not working.} The two other terms are the sum of skill costs of job $j$'s constituent
modules (AI chains and manual tasks). Similarly, the time (or labor)
cost of job $j$ is:
\begin{equation}
t_{j}:=t_{j}^{s}+\sum_{a\in\mathcal{A}_{j}}t_{a}^{AI}\alpha^{-d_{a}}+\sum_{h\in\mathcal{H}_{j}}t_{h}^{H}.\label{eq:time}
\end{equation}
\item The wage bill for job $j$ is the product of $w_{j}$ and $t_{j}$:
\begin{equation}
WageBill_{j}=w_{j}t_{j},\label{eq:wage_bill}
\end{equation}
which represents workers being compensated proportionally to their
skill level and the time required to complete the job. Producing one
unit of the consumption good $X$ only involves paying the labor time
cost of workers performing tasks directly associated with that unit,
as the one-time human capital training costs have already been incurred.
\item With this production function, optimal allocation of tasks to jobs
(and thus job boundaries) are determined by solving firm's cost minimization
problem:
\[
\min_{J,\mathcal{J}=\{\mathcal{H}_{j},\mathcal{A}_{j}\}_{j=1}^{J}}\ \ \sum_{j=1}^{J}w_{j}t_{j}=\sum_{j=1}^{J}\left[\left(v_{L}+\sum_{a\in\mathcal{A}_{j}}c_{a}^{AI}+\sum_{h\in\mathcal{H}_{j}}c_{h}^{H}\right)\left(t_{j}^{s}+\sum_{a\in\mathcal{A}_{j}}t_{a}^{AI}\alpha^{-d_{a}}+\sum_{h\in\mathcal{H}_{j}}t_{h}^{H}\right)\right].
\]
(We have a section dedicated to solving this problem via dynamic programming.)
\end{itemize}

\section{General Equilibrium}

The AI's success probability in completing AI module $a\in\mathcal{A}$,
denoted by $q_{a}$, in every attempt is
\[
q_{a}=\alpha^{d_{a}},
\]
where $\alpha\in(0,1)$ is the (general) AI quality and $d_{a}>0$
is the module's completion difficulty. The AI's (expected) time cost
to complete module $a$ is:
\[
\frac{t_{a}^{AI}}{q_{a}}=\frac{t_{a}^{AI}}{\alpha^{d_{a}}}=t_{a}^{AI}\alpha^{-d_{a}},
\]
where $t_{a}^{AI}$ is the management cost of AI module $a$.

Each task is either done manually by a human or is a part of an AI-completed
chain. If the task $i$ is done manually, its time cost is simply
just $t_{i}^{H}$. If task $i$ is a part of an AI chain $a$ it is
either augmented or automated. The success probability of any task
of an AI module $a$ appears as a multiplicative term in the success
probability of the chain $q_{a}$. If task $i$ is augmented, it management
time $t_{i}^{AI}$ also determines AI chain's management time whereas
if automated task's contribution to the AI chain cost is limited to
its success probability alone.


\subsection{Production}

There are $K$ consumption goods $X_{1},\cdots,X_{K}$ in the economy.
We call each $k\in\{1,\cdots,K\}$ a sector. A single firm produces
all goods and sells them to the consumers in a perfectly competitive
market. 

The production function for consumption good $X_{k}$ is CES with
three components (human labor, AI labor, AI quality) modeled as two
separate inputs (human and machine):
\begin{equation}
X_{k}=\left(\sum_{j=1}^{J_{k}}\left[\sum_{a_{j}\in\mathcal{A}_{j}^{k}}\eta_{a_{j}}^{k}\left(t_{a_{j}}^{AI,k}\alpha^{-d_{a_{j}}}\right)^{\rho}+\sum_{h_{j}\in\mathcal{H}_{j}^{k}}\eta_{h_{j}}\left(t_{h_{j}}^{H,k}\right)^{\rho}\right]\right)^{\frac{1}{\rho}}\label{eq:prod_fun}
\end{equation}
Here, $t_{a_{j}}^{AI,k}$ is the AI management time cost for composite
AI chain $a_{j}$ and $t_{h_{j}}^{H,k}$ is the human execution time
cost for composite manual task $h_{j}$. Note that the production
function exhibits constant elasticity of substitution $\sigma=\frac{1}{1-\rho}$
between all modules (AI chains and manual tasks).

For some fixed job design $\mathcal{J}^{k}=\{\mathcal{J}_{j}^{k}\}_{j=1}^{J_{k}}=\{\mathcal{H}_{j}^{k},\mathcal{A}_{j}^{k}\}_{j=1}^{J_{k}}$,
The cost of producing $X$ units is:
\begin{equation}
C_{k}=\sum_{j=1}^{J_{k}}w_{j}^{k}t_{j}^{k}=\sum_{j=1}^{J_{k}}\left[\underbrace{\left(v_{L}+\sum_{a_{j}\in\mathcal{A}_{j}^{k}}c_{a_{j}}^{AI,k}+\sum_{h_{j}\in\mathcal{H}_{j}^{k}}c_{h_{j}}^{AI,k}\right)}_{\text{wage of job \ensuremath{j}: }w_{j}^{k}}\underbrace{\left(t_{j}^{s}+\sum_{a_{j}\in\mathcal{A}_{j}^{k}}t_{a_{j}}^{AI,k}\alpha^{-d_{a}}+\sum_{h_{j}\in\mathcal{H}_{j}^{k}}t_{h_{j}}^{AI,k}\right)}_{\text{amount of labor employed for job \ensuremath{j}}}\right].\label{eq:firm_cost}
\end{equation}
The wage of AI chain input and human input in manual task is the same
and equal to $w_{j}=\left(v_{L}+\sum_{a_{j}\in\mathcal{A}_{j}^{k}}c_{a_{j}}^{AI,k}+\sum_{h_{j}\in\mathcal{H}_{j}^{k}}c_{h_{j}}^{AI,k}\right)$
of job $j$ containing those inputs. 

The firm's profit maximization problem in any sector $k$ is:
\[
\max_{\{t_{h_{j}}^{H,k}\},\{t_{a_{j}}^{AI,k}\},\{\mathcal{H}_{j}^{k},\mathcal{A}_{j}^{k}\}_{j=1}^{J_{k}},J_{k}}\ \ p^{k}X_{k}-C_{k},
\]
which in expanded form becomes:
\begin{align}
\max_{\{t_{h_{j}}^{H,k}\},\{t_{a_{j}}^{AI,k}\},\{\mathcal{H}_{j}^{k},\mathcal{A}_{j}^{k}\}_{j=1}^{J_{k}},J_{k}}\ \  & p^{k}\left(\sum_{j=1}^{J_{k}}\left[\sum_{a_{j}\in\mathcal{A}_{j}^{k}}\eta_{a_{j}}^{k}\left(t_{a_{j}}^{AI,k}\alpha^{-d_{a_{j}}}\right)^{\rho}+\sum_{h_{j}\in\mathcal{H}_{j}^{k}}\eta_{h_{j}}\left(t_{h_{j}}^{H,k}\right)^{\rho}\right]\right)^{\frac{1}{\rho}}\label{eq:firm_problem}\\
 & \ \ \ \ -\sum_{j=1}^{J_{k}}\left[w_{j}^{k}\left(t_{j}^{s}+\sum_{a_{j}\in\mathcal{A}_{j}^{k}}t_{a_{j}}^{AI,k}\alpha^{-d_{a}}+\sum_{h_{j}\in\mathcal{H}_{j}^{k}}t_{h_{j}}^{AI,k}\right)\right]
\end{align}
The firm chooses which tasks to do manually and which ones via AI,
the amount of labor employed for manual tasks $t_{h_{j}}^{H,k}$,
amount of labor employed for managing AI chains $t_{a_{j}}^{AI,k}$,
and the job assignments $\mathcal{J}^{k}=\{\mathcal{H}_{j}^{k},\mathcal{C}_{j}^{k}\}_{j=1}^{J_{k}}$
while taking the AI quality $\alpha$ as given. 

(?)This problem can be solved in two steps. In the (discrete) outer
loop we fix some job assignment and in the (continuous) inner loop
we solve for optimal labor allocations given a job design. The solution
will be the job design maximizing profits.

Equivalently, we can consider the dual problem:
\begin{align}
\min_{\{t_{h}^{H}\},\{t_{a}^{AI}\},\{\mathcal{H}_{j},\mathcal{A}_{j}\}_{j=1}^{J},J}\  & C_{k}=\sum_{j=1}^{J}\left[w_{j}\left(t_{j}^{s}+\sum_{a\in\mathcal{A}_{j}}t_{a}^{AI}\alpha^{-d_{a}}+\sum_{h\in\mathcal{H}_{j}}t_{h}^{H}\right)\right]\label{eq:dual}\\
s.t.\  & X_{k}=\left(\sum_{j=1}^{J}\left[\sum_{a\in\mathcal{A}_{j}^{k}}\eta_{a}\left(t_{a}^{AI}\alpha^{-d_{a}}\right)^{\rho}+\sum_{h\in\mathcal{H}_{j}}\eta_{h}\left(t_{h}^{H}\right)^{\rho}\right]\right)^{\frac{1}{\rho}}\nonumber 
\end{align}


\subsection{Consumption}

A representative household supplies one unit of labor inelastically
and has a CES utility function over the $K$ goods:
\[
U(X_{1},...,X_{K})=\left(\sum_{k=1}^{K}\delta_{k}X_{k}^{\frac{\sigma-1}{\sigma}}\right)^{\frac{\sigma}{\sigma-1}},\qquad0<\delta_{k}<1,\qquad\sum_{k=1}^{K}\delta_{k}=1,\qquad\sigma>0.
\]
Goods are substitutes if $\sigma>1$, complements if $\sigma<1$,
and $\sigma=1$ corresponds to the Cobb--Douglas case. The household
faces the budget constraint
\begin{equation}
\sum_{k=1}^{K}p^{k}X_{k}=\sum_{k=1}^{K}\left(\sum_{j=1}^{J_{k}}w_{j}^{k}t_{j}^{k}\right),\label{eq:budget_constraint}
\end{equation}
where $I$ is the total income earned from supplying labor to jobs
across all sectors.

\subsection{Labor Market Clearing}

The (representative) household supplies one unit of labor inelastically
to all modules across all sectors. The labor market clearing condition
satisfies:
\begin{equation}
\sum_{k=1}^{K}\left(\sum_{j=1}^{J_{k}}\left[\sum_{a\in\mathcal{A}_{j}^{k}}t_{a}^{AI,k}+\sum_{h\in\mathcal{H}_{j}^{k}}t_{h}^{H,k}\right]\right)=1.\label{eq:labor_market_clearing}
\end{equation}


\subsection{Some Math}

\subsubsection{Production Side Equations}

First, notice that since markets are competitive price of good $k$,
$p^{k}$, is given by:
\begin{equation}
p^{k}=\frac{C_{k}}{X_{k}}\label{eq:price}
\end{equation}
Suppose the job design is fixed. In the inner loop firm solves (\ref{eq:firm_problem})
via FOCs:
\begin{align*}
[t_{a}^{AI}] & :p^{k}X^{1-\rho}\eta_{a}\rho(t_{a_{j}}^{AI,k}\alpha^{-d_{a_{j}}})^{\rho-1}\alpha^{-d_{a_{j}}}=w_{j}\alpha^{-d_{a}}\quad\Rightarrow\quad t_{a_{j}}^{AI,k}=\alpha^{d_{a_{j}}}\left(\frac{w_{j}}{p^{k}\rho\eta_{a}X^{1-\rho}}\right)^{\frac{1}{\rho-1}}\\{}
[t_{h}^{H}] & :p^{k}X^{1-\rho}\eta_{h}\rho(t_{h_{j}}^{H,k})^{\rho-1}=w_{j}\quad\Rightarrow\quad t_{h_{j}}^{H,k}=\left(\frac{w_{j}}{p^{k}\rho\eta_{h}X^{1-\rho}}\right)^{\frac{1}{\rho-1}}
\end{align*}
And thus the ratio of AI and human labor in a job is:
\begin{equation}
\frac{t_{a}^{AI,k}}{t_{h}^{H,k}}=\alpha^{d_{a}}\left(\frac{\eta_{h}}{\eta_{a}}\right)^{\frac{1}{\rho-1}},\label{eq:input_ratio}
\end{equation}
while the ratio of AI labor between two jobs is:
\begin{equation}
\frac{t_{a_{j}}^{AI,k}}{t_{a_{j^{'}}}^{AI,k}}=\alpha^{d_{a_{j}}-d_{a_{j^{'}}}}\left(\frac{w_{j}}{w_{j^{'}}}\right)^{\frac{1}{\rho-1}}.\label{eq:cross_job_AI_ratio}
\end{equation}
Equation (\ref{eq:cross_job_AI_ratio}) implies that job $j$ will
have more AI labor in equilibrium if: 1) it is more ``AI-able'',
meaning that AI is on average more successful in executing AI chains
in job $j$, and 2) job $j$ is cheaper than job $j^{'}$.  

Substituting (\ref{eq:input_ratio}) into the labor market clearing
(\ref{eq:labor_market_clearing}) determines the level of human labor
demanded:
\[
\sum_{k=1}^{K}\sum_{j=1}^{J_{k}}\left[\sum_{h\in\mathcal{H}_{j}^{k}}t_{h}^{H,k}+\sum_{a\in\mathcal{A}_{j}^{k}}t_{h}^{H,k}\,\alpha^{d_{a}}\left(\frac{\eta_{h}}{\eta_{a}}\right)^{\frac{1}{\rho-1}}\right]=1.
\]
Finally, using (\ref{eq:price}) and (\ref{eq:input_ratio}) the price
of good $k$ can be written as a function of human labor demanded
in each job:
\[
p^{k}=\frac{\sum_{j=1}^{J_{k}}w_{j}^{k}\left[t_{j}^{s}+\sum_{h_{j}\in\mathcal{H}j^{k}}t{}_{h_{j}}^{H,k}\left(1+\alpha^{d_{a_{j}}}\left(\frac{\eta_{h_{j}}}{\eta_{a_{j}}}\right)^{\frac{1}{\rho-1}}\alpha^{-d_{a_{j}}}\right)\right]}{\left(\sum_{j=1}^{J_{k}}\sum_{h_{j}\in\mathcal{H}j^{k}}\eta_{h_{j}}(t_{h_{j}}^{H,k})^{\rho}\left[1+\left(\frac{\eta_{a_{j}}}{\eta_{h_{j}}}\right)^{\frac{-1}{\rho-1}}\right]\right)^{\frac{1}{\rho}}}.
\]


\subsubsection{Supply Side Equations}

From the household's problem we obtain the demand for consumption
good $X_{k}$ as:
\begin{equation}
X_{k}(p^{1},\ldots,p^{K},I)=\frac{\delta_{k}^{\sigma}(p^{k})^{-\sigma}}{\sum_{l=1}^{K}\delta_{l}^{\sigma}(p^{l})^{1-\sigma}}\left(\sum_{k\text{\textquoteright}=1}^{K}\sum_{j=1}^{J_{k\text{\textquoteright}}}w_{j}^{k\text{\textquoteright}}\left(t_{j}^{s,k\text{\textquoteright}}+\sum_{a_{j}\in\mathcal{A}_{j}^{k\text{\textquoteright}}}t_{a_{j}}^{AI,k\text{\textquoteright}}\alpha^{-d_{a}}+\sum_{h_{j}\in\mathcal{H}_{j}^{k\text{\textquoteright}}}t_{h_{j}}^{AI,k\text{\textquoteright}}\right)\right).\label{eq:good_demand}
\end{equation}


\subsection{Equilibrium}

Equilibrium is characterized by the following 6 equations that must
be solved simultaneously. The unknowns are $\{\{t_{h_{j}}^{H,k}\}_{j=1}^{J_{k}}\}_{k=1}^{K}$,
$\{\{t_{a_{j}}^{AI,k}\}_{j=1}^{J_{k}}\}_{k=1}^{K}$, $\{X_{k}\}_{k=1}^{K}$,
and $\{p^{k}\}_{k=1}^{K}$:
\begin{enumerate}
\item Optimal labor allocation:
\[
t_{h_{j}}^{H,k}=\left(\frac{w_{j}}{p^{k}\rho\eta_{h}X^{1-\rho}}\right)^{\frac{1}{\rho-1}},\qquad t_{a_{j}}^{AI,k}=\alpha^{d_{a_{j}}}\left(\frac{w_{j}}{p^{k}\rho\eta_{a}X^{1-\rho}}\right)^{\frac{1}{\rho-1}}.
\]
\item Supply and demand of goods:
\begin{align*}
X_{k}^{S} & =\left(\sum_{j=1}^{J_{k}}\left[\sum_{a_{j}\in\mathcal{A}_{j}^{k}}\eta_{a_{j}}^{k}\left(t_{a_{j}}^{AI,k}\alpha^{-d_{a_{j}}}\right)^{\rho}+\sum_{h_{j}\in\mathcal{H}_{j}^{k}}\eta_{h_{j}}\left(t_{h_{j}}^{H,k}\right)^{\rho}\right]\right)^{\frac{1}{\rho}},\\
X_{k}^{D} & =\frac{\delta_{k}^{\sigma}(p^{k})^{-\sigma}}{\sum_{l=1}^{K}\delta_{l}^{\sigma}(p^{l})^{1-\sigma}}\left(\sum_{k\text{\textquoteright}=1}^{K}\sum_{j=1}^{J_{k\text{\textquoteright}}}w_{j}^{k\text{\textquoteright}}\left(t_{j}^{s,k\text{\textquoteright}}+\sum_{a_{j}\in\mathcal{A}_{j}^{k\text{\textquoteright}}}t_{a_{j}}^{AI,k\text{\textquoteright}}\alpha^{-d_{a}}+\sum_{h_{j}\in\mathcal{H}_{j}^{k\text{\textquoteright}}}t_{h_{j}}^{AI,k\text{\textquoteright}}\right)\right),\\
X_{k}^{D} & =X_{k}^{S}.
\end{align*}
\item Prices of goods follow:
\[
p^{k}=\frac{\sum_{j=1}^{J_{k}}w_{j}^{k}\left[t_{j}^{s}+\sum_{h_{j}\in\mathcal{H}j^{k}}t_{h_{j}}^{H,k}+\sum_{a_{j}\in\mathcal{A}j^{k}}t_{a_{j}}^{AI,k}\alpha^{-d_{a_{j}}}\right]}{X_{k}}.
\]
\item Labor market clears:
\[
\sum_{k=1}^{K}\sum_{j=1}^{J_{k}}\left[\sum_{h\in\mathcal{H}j^{k}}t_{h_{j}}^{H,k}+\sum_{a\in\mathcal{A}j^{k}}t_{a_{j}}^{AI,k}\right]=1.
\]
\end{enumerate}
Overall, we have one more equation than we have unknowns. So, we can
solve for the equilibrium allocations and prices numerically. Note
that these solutions take the job design as fixed. In the next step
we should choose the optimal job design in the outer loop.
\end{document}
