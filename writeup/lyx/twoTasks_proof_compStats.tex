%% LyX 2.4.2.1 created this file.  For more info, see https://www.lyx.org/.
%% Do not edit unless you really know what you are doing.
\documentclass[american]{article}
\usepackage[T1]{fontenc}
\usepackage[utf8]{inputenc}
\usepackage{amsmath}
\usepackage{amsthm}
\usepackage{geometry}
\geometry{verbose,tmargin=2cm,bmargin=2cm,lmargin=1.75cm,rmargin=1.75cm}
\PassOptionsToPackage{normalem}{ulem}
\usepackage{ulem}

\makeatletter
%%%%%%%%%%%%%%%%%%%%%%%%%%%%%% Textclass specific LaTeX commands.
\theoremstyle{plain}
    \ifx\thechapter\undefined
      \newtheorem{lem}{\protect\lemmaname}
    \else
      \newtheorem{lem}{\protect\lemmaname}[chapter]
    \fi
\theoremstyle{plain}
    \ifx\thechapter\undefined
  \newtheorem{cor}{\protect\corollaryname}
\else
      \newtheorem{cor}{\protect\corollaryname}[chapter]
    \fi

\makeatother

\usepackage{babel}
\providecommand{\corollaryname}{Corollary}
\providecommand{\lemmaname}{Lemma}

\begin{document}
In this appendix, we analyze the properties of the optimal paths for
the dynamic programming problem specified in {[}{]}. The solution
follows this structure: allocate all resources to the larger of $q_{1}$
and $q_{2}$ until a specific point, then switch to investing all
resources in the other task’s capability until parity between $q_{1}$
and $q_{2}$ is achieved. After reaching parity, resources are evenly
distributed between both tasks. While we are unable to analytically
determine the exact values of $q_{1}$ and $q_{2}$ at the turning
point, we provide comparative statics with respect to the initial
values and the discount rate.

We reformulate the problem as a maximization problem:
\begin{align*}
\max_{q_{1}(t),q_{2}(t)} & \int_{0}^{T}-e^{-\rho t}C(q_{1},q_{2})dt\\
s.t. & \frac{dq_{1}}{dt}+\frac{dq_{2}}{dt}+s(t)=r,\\
 & q_{1},q_{2}\geq0,\\
 & q_{1}(0)=q_{1,0},q_{2}(0)=q_{2,0}.
\end{align*}
Here, $q_{1},q_{2}$ are the state variables, while $\frac{dq_{1}}{dt}$
and $\frac{dq_{2}}{dt}$ are control variables. We form the Hamiltonian
as follows: 
\[
H=-e^{-\rho t}C(q_{1},q_{2})+\lambda_{1}\frac{dq_{1}}{dt}+\lambda_{2}\frac{dq_{2}}{dt}+\mu\left(r-\frac{dq_{1}}{dt}-\frac{dq_{2}}{dt}\right),
\]
where $\lambda_{1},\lambda_{2}$ are costate variables, and $\mu$
is the Lagrange multiplier associated with the investment constraint.
The necessary conditions for an \uline{interior} solution are:
\begin{align*}
\frac{dq_{i}}{dt} & =\frac{\partial H}{\partial\lambda_{i}},\tag{State Dynamics}\\
\frac{d\lambda_{i}}{dt} & =-\frac{\partial H}{\partial q_{i}},\tag{Costate Dynamics}\\
\frac{\partial H}{\partial\left(\frac{dq_{i}}{dt}\right)} & =0,\tag{Stationarity}\\
\mu\left(r-\frac{dq_{1}}{dt}-\frac{dq_{2}}{dt}-s\right) & =0.\tag{Complementary Slackness}
\end{align*}
Expanding the formulas, we obtain:
\begin{align*}
\frac{dq_{1}}{dt} & =\frac{dq_{1}}{dt},\\
\frac{dq_{2}}{dt} & =\frac{dq_{2}}{dt},\\
\frac{d\lambda_{1}}{dt} & =e^{-\rho t}\frac{\partial C}{\partial q_{1}},\\
\frac{d\lambda_{2}}{dt} & =e^{-\rho t}\frac{\partial C}{\partial q_{2}},\\
\lambda_{1}-\mu & =0,\\
\lambda_{2}-\mu & =0,\\
\mu>0\ and\ \frac{dq_{1}}{dt}+\frac{dq_{2}}{dt}=r, & \ or\ \mu=0\ and\ \frac{dq_{1}}{dt}+\frac{dq_{2}}{dt}<r.
\end{align*}
The costate dynamics imply
\[
\frac{\frac{d\lambda_{1}}{dt}}{\frac{d\lambda_{2}}{dt}}=\frac{\frac{\partial C}{\partial q_{1}}}{\frac{\partial C}{\partial q_{2}}},
\]
and from the stationarity conditions, we have:
\[
\lambda_{1}=\lambda_{2}=\mu.
\]

The $\mu=0$ case can happen only if:
\[
\frac{\partial C}{\partial q_{1}}=\frac{\partial C}{\partial q_{2}}=0
\]
for all times. In this scenario, any $q_{1}$ and $q_{2}$ would suffice
since neither contributes to the objective function. However, this
is not applicable in our setup because, as soon as task $i$ becomes
automated, the corresponding partial derivative changes from 0 to
$-\frac{cm}{q_{i}^{2}}$.\footnote{We assume $c_{m}$ and $c_{h}$ are such that some level of automation
is preferred over no automation. Otherwise, the solution would be
trivial, and no tasks would be automated.} Therefore, we must have $\mu>0$, and the investment endowment is
always fully utilized.

\uline{If} the solution is interior, then 
\[
\lambda_{1}=\lambda_{2}=\mu>0
\]
and the following holds for all $t$:
\[
\bar{\lambda}_{1}+\int_{0}^{t}e^{-\rho s}\frac{\partial C}{\partial q_{1}}ds=\bar{\lambda}_{2}+\int_{0}^{t}e^{-\rho s}\frac{\partial C}{\partial q_{2}}ds
\]
Assume that at $t=T$ (the horizon), $\lambda_{1}=\lambda_{2}>0$.\footnote{Intuitively, this means that at the horizon, we stop investing in
either variable, resulting in a “steady state.” This is reasonable
because, once $q=1$, no further investments are required.} This implies at the optimum, the marginal cost with respect to task
capabilities must be equal. Thus:
\[
\forall t:\frac{\partial C}{\partial q_{1}}=\frac{\partial C}{\partial q_{2}}
\]
Given the functional form of the cost function, such scenario can
occur only when both tasks are automated, \uline{and} $q_{1}=q_{2}$.
Specifically:
\[
\frac{\partial C\{<1|2>\}}{\partial q_{1}}=\frac{\partial C\{<1|2>\}}{\partial q_{2}}\Leftrightarrow-\frac{c_{m}}{q_{1}^{2}q_{2}}=-\frac{c_{m}}{q_{1}q_{2}^{2}}\Leftrightarrow q_{1}=q_{2}.
\]
Unless this condition is met, we have a corner solution where all
the available endowment is allocated to improving only one task’s
AI capability. 

We \uline{conjecture} that at a sufficiently long horizon, $q_{1}=q_{2}$,
and proceed to determine the optimal trajectories of $q_{1}$ and
$q_{2}$ using backward induction. Now, imagine we are in the <1|2>
region. Let $\tilde{T}$ denote the earliest time at which $\frac{\partial C}{\partial q_{1}}=\frac{\partial C}{\partial q_{2}}$
holds.\footnote{We assume that the parameters c\_\{m\} and c\_\{h\} are such that
this equality is guaranteed to occur. \uline{Otherwise, the solution
would trivially involve investing in only one q, fully maximizing
it before switching to the other until it too is maximized.}} Formally: 
\[
\tilde{T}=inf\{t:\frac{\partial C}{\partial q_{1}}=\frac{\partial C}{\partial q_{2}}\}.
\]
For $t<\tilde{T}$, we have a corner solution. Without loss of generality,
assume that $\frac{\partial C}{\partial q_{1}}>\frac{\partial C}{\partial q_{2}}$
when $t<\tilde{T}$. With a slight abuse of notation, consider a time
$\delta$ arbitrarily close to and just before $\tilde{T}$. At such
times, we can write:
\[
\frac{\partial C\{<1|2>\}}{\partial q_{1}}\approx-\frac{c_{m}}{q_{1}^{2}q_{2}}>-\frac{c_{m}}{q_{1}q_{2}^{2}}\approx\frac{\partial C\{<1|2>\}}{\partial q_{2}}.
\]
This simplifies to:
\[
-\frac{1}{q_{1}}>-\frac{1}{q_{2}}\Leftrightarrow q_{1}>q_{2}.
\]
 Therefore, for times $t<\tilde{T}$, we must invest exclusively in
$q_{2}$ until parity between $q_{1}$ and $q_{2}$ is achieved. This
also implies that we are in the region below the 45-degree line and
are moving toward it.

\bigskip{}

Now, we continue going backward until we reach either the boundary
of $<1><2>$ or $<1>(2)$. Here we only consider the case where we
hit the boundary of $<1><2>$.\footnote{The dynamics of transitioning directly from <1|2> to <1>(2) are less
complicated than transitioning through <1><2> first and then to <1>(2).
For brevity, we do not describe the direct transition separately,
as the mechanism is similar and requires one less step of analysis.} Given that we are in the region where $q_{1}>q_{2}$, the following
conditions hold:
\[
q_{1}>q_{2}\Leftrightarrow-\frac{c_{m}}{q_{1}^{2}}>-\frac{c_{m}}{q_{2}^{2}}\Leftrightarrow\frac{\partial C\{<1><2>\}}{\partial q_{1}}>\frac{\partial C\{<1><2>\}}{\partial q_{2}}.
\]
Thus, in the <1><2> region, we must continue investing in $q_{2}$.
This remains true even if we enter <1><2> from the point of parity
in <1|2>. In other words, in the moments leading up to $\frac{\partial C}{\partial q_{1}}=\frac{\partial C}{\partial q_{2}}$,
we will allocate all resources to $q_{2}$, regardless of whether
we will be in <1|2> or <1><2> at $\tilde{T}$.

Now imagine transitioning from <1><2> to <1>(2). The cost function
changes from:
\[
C\{<1><2>\}=\frac{c_{m}}{q_{1}}+\frac{c_{m}}{q_{2}}
\]
to 
\[
C\{<1>(2)\}=\frac{c_{m}}{q_{1}}+c_{h}.
\]
Let us maintain the assumption that $q_{1}>q_{2}$ as we cross the
horizontal boundary in Figure 3 and enter the new region. In the <1><2>
region, we have: $\frac{\partial C}{\partial q_{1}}>\frac{\partial C}{\partial q_{2}}$.
However, on the <1>(2) side, the derivatives change to $\frac{\partial C}{\partial q_{1}}=-\frac{c_{m}}{q_{1}^{2}}<0$
and $\frac{\partial C}{\partial q_{2}}=0$. This implies that the
task contributing more to the objective function differs across these
regions. However, this does not mean the optimal investment path switches
instantaneously. While the instantaneous benefits shift at the boundary,
the agent must account for the continuation value of investments in
future periods. This is because, despite the discontinuity in the
objective function, the \uline{state variables change continuously}.

Thus, when transitioning from <1><2> to <1>(2), we continue investing
in $q_{2}$ even though $q_{1}$ provides a higher instantaneous utility.
We maintain this allocation until reaching a point where the continuation
value of investing in $q_{2}$ equals the benefit of switching to
$q_{1}$. This point represents a kink in the optimal trajectory of
the state variables. Let $\hat{T}$ denote the time at which we reach
this kink. At $\hat{T}$, the following condition must hold:
\[
\underbrace{\int_{\hat{T}}^{\tau}e^{-\rho s}(\frac{c_{m}}{\hat{q}_{1}}+c_{h})ds}_{\text{from \ensuremath{\hat{T}}to when automate task 2}}+\underbrace{\int_{\tau}^{\tilde{T}}e^{-\rho s}(\frac{c_{m}}{\hat{q}_{1}(q_{2,0}+rs)})ds}_{\text{from automation of task 2 to parity of \ensuremath{q}s}}+\underbrace{\int_{\tilde{T}}^{\bar{T}}e^{-\rho s}(\frac{c_{m}}{\hat{q}_{1}^{2}})ds}_{\text{after parity}}=\underbrace{\int_{\hat{T}}^{\bar{T}}e^{-\rho s}(\frac{c_{m}}{\hat{q}_{1}+rs}+c_{h})ds}_{\text{continue investing on \ensuremath{q_{1}}}}.\tag{Kink Equation}
\]
Here, $\hat{q}_{1}$ denotes the value of $q_{1}$ at the kink, $\tau$
is the time we cross the boundary to automating task 2, and $\tilde{T}$
is the time when parity between $q_{1}$ and $q_{2}$ is achieved.
For $t<\hat{T}$, the right-hand side (RHS) of the equation is smaller
than the left-hand side (LHS). Thus, the agent continues investing
in $q_{1}$ until the two sides are equalized. This condition holds
regardless of whether we are in the <1>(2) region or cross the boundary
to (1)(2). Specifically: 
\[
\frac{\partial C\{(1)(2)\}}{\partial q_{1}}=\frac{\partial C\{(1)(2)\}}{\partial q_{2}}=0,
\]
whereas:
\[
\frac{\partial C\{<1>(2)\}}{\partial q_{1}}=-\frac{c_{m}}{q_{1}^{2}}<0=\frac{\partial C\{<1>(2)\}}{\partial q_{2}}.
\]
This completes the characterization of the optimal trajectory of the
state variables, starting from any point on the $q_{1}$-$q_{2}$
grid.

\section*{Comparative Statics}

The following lemmas provide comparative statics of optimal trajectories
with respect to $q_{2,0}$ and $\rho$. We continue to maintain the
assumption that we are below the 45 degree line in the $q_{1}-q_{2}$
plane (i.e., $q_{1}>q_{2}$ region). 
\begin{lem}
$\frac{\partial\hat{T}}{\partial q_{2,0}}<0$.
\end{lem}
\begin{proof}
We have:
\begin{align*}
\frac{\partial\hat{T}}{\partial q_{2,0}} & =-\int_{\tau}^{\tilde{T}}e^{-\rho s}(\frac{c_{m}}{\hat{q}_{1}(q_{2,0}+rs)^{2}})ds<0.
\end{align*}

\noindent Lemma 1 states that if the initial value of $q_{2,0}$ increases,
the switch from investing in $q_{1}$ to $q_{2}$ occurs earlier.
This explains the shift in kinks of the optimal paths in Figure 4---moving
from light blue to green to dark blue---as they shift upward and
to the left.
\end{proof}
\begin{lem}
$\frac{\partial\hat{q}_{1}}{\partial q_{2,0}}<0$.
\end{lem}
\begin{proof}
Note that $\frac{\partial\hat{q}_{1}}{\partial\hat{T}}=r$, because
$\hat{T}$ is defined as the time during which all resources ($r$)
are allocated exclusively to $q{1}$ for $t<\hat{T}$. Using this,
we can write:
\begin{align*}
\frac{\partial\hat{q}_{1}}{\partial q_{2,0}} & =\frac{\partial\hat{q}_{1}}{\partial\hat{T}}\frac{\partial\hat{T}}{\partial q_{2,0}}\\
 & =r\times\frac{\partial\hat{T}}{\partial q_{2,0}}\\
 & <0,
\end{align*}
where the last line follows from Lemma 1.
\end{proof}
\begin{lem}
A sufficient condition for $\frac{\partial\hat{T}}{\partial\rho}<0$
is $\frac{c_{m}}{c_{h}}>Q(\hat{q}_{1})$, where $Q(\hat{q}_{1})=\frac{\hat{q}_{1}^{2}}{1-\hat{q}_{1}}$,
the machine-to-human comparative advantage bound, is increasing in
$\hat{q}_{1}$. 
\end{lem}
\begin{proof}
We take partial derivative of both sides of the Kink equation {[}{]}
with respect to $\rho$, treating $\hat{T}$ as a variable:

\[
\int_{\hat{T}}^{\tau}e^{-\rho s}(\frac{c_{m}}{\hat{q}_{1}}+c_{h})ds+\int_{\tau}^{\tilde{T}}e^{-\rho s}(\frac{c_{m}}{\hat{q}_{1}(q_{2,0}+rs)})ds+\int_{\tilde{T}}^{\bar{T}}e^{-\rho s}(\frac{c_{m}}{\hat{q}_{1}^{2}})ds=\int_{\hat{T}}^{\bar{T}}e^{-\rho s}(\frac{c_{m}}{\hat{q}_{1}+rs}+c_{h})ds
\]
Differentiating with respect to $\rho$ gives: 
\begin{align*}
-\int_{\hat{T}}^{\tau}se^{-\rho s}(\frac{c_{m}}{\hat{q}_{1}}+c_{h})ds-\int_{\tau}^{\tilde{T}}se^{-\rho s}(\frac{c_{m}}{\hat{q}_{1}(q_{2,0}+rs)})ds\\
-\int_{\tilde{T}}^{\bar{T}}se^{-\rho s}(\frac{c_{m}}{\hat{q}_{1}^{2}})ds-\left[\frac{\partial\hat{T}}{\partial\rho}e^{-\rho\hat{T}}(\frac{c_{m}}{\hat{q}_{1}}+c_{h})\right] & =-\int_{\hat{T}}^{\bar{T}}se^{-\rho s}(\frac{c_{m}}{\hat{q}_{1}+rs}+c_{h})ds-\left[\frac{\partial\hat{T}}{\partial\rho}e^{-\rho\hat{T}}\left(\frac{c_{m}}{\hat{q}_{1}+rt}+c_{h}\right)\right]
\end{align*}
Rearrange terms, we get: 
\begin{align*}
\frac{\partial\hat{T}}{\partial\rho} & =\frac{\int_{\hat{T}}^{\tau}se^{-\rho s}(\frac{c_{m}}{\hat{q}_{1}}+c_{h})ds+\int_{\tau}^{\tilde{T}}se^{-\rho s}(\frac{c_{m}}{\hat{q}_{1}(q_{2,0}+rs)})ds+\int_{\tilde{T}}^{\bar{T}}se^{-\rho s}(\frac{c_{m}}{\hat{q}_{1}^{2}})ds-\int_{\hat{T}}^{\bar{T}}se^{-\rho s}(\frac{c_{m}}{\hat{q}_{1}+rs}+c_{h})ds}{e^{-\rho\hat{T}}\left(\frac{c_{m}}{\hat{q}_{1}+rt}-\frac{c_{m}}{\hat{q}_{1}}\right)}
\end{align*}
Since the denominator is always negative ($\hat{q}_{1}+rt>\hat{q}_{1}$),
the sign of $\frac{\partial\hat{T}}{\partial\rho}$ depends on the
numerator. Breaking the last integral into three intervals:
\begin{align*}
\int_{\hat{T}}^{\tau}se^{-\rho s}(\frac{c_{m}}{\hat{q}_{1}}+c_{h})ds+\int_{\tau}^{\tilde{T}}se^{-\rho s}(\frac{c_{m}}{\hat{q}_{1}(q_{2,0}+rs)})ds+\int_{\tilde{T}}^{\bar{T}}se^{-\rho s}(\frac{c_{m}}{\hat{q}_{1}^{2}})ds-\int_{\hat{T}}^{\bar{T}}se^{-\rho s}(\frac{c_{m}}{\hat{q}_{1}+rs}+c_{h})ds\\
=\underbrace{\int_{\hat{T}}^{\tau}se^{-\rho s}(\frac{c_{m}}{\hat{q}_{1}}-\frac{c_{m}}{\hat{q}_{1}+rs})ds}_{>0}+\int_{\tau}^{\tilde{T}}se^{-\rho s}(\frac{c_{m}}{\hat{q}_{1}(q_{2,0}+rs)}-\frac{c_{m}}{\hat{q}_{1}+rs}-c_{h})ds+\int_{\tilde{T}}^{\bar{T}}se^{-\rho s}(\frac{c_{m}}{\hat{q}_{1}^{2}}-\frac{c_{m}}{\hat{q}_{1}+rs}-c_{h})ds
\end{align*}
Now we look for sufficient conditions that guarantee the numerator
is positive. First, notice that $\left(-\frac{c_{m}}{\hat{q}_{1}+rt}-c_{h}\right)$
appears in both the middle and right terms. To ensure the numerator
is positive, we require:
\[
\frac{c_{m}}{\hat{q}_{1}+rt}+c_{h}\leq min\{\frac{c_{m}}{\hat{q}_{1}^{2}},\frac{c_{m}}{\hat{q}_{1}(q_{2,0}+rt)}\}
\]
In the region where $q_{1}>q_{2}$ and $q_{2}+rt\leq\hat{q}_{1}$
until parity is achieved, the minimum is $\frac{c_{m}}{\hat{q}{1}^{2}}$.
Therefore, the condition becomes: 
\begin{align*}
\frac{c_{m}}{\hat{q}_{1}+rt}+c_{h} & <\frac{c_{m}}{\hat{q}_{1}^{2}}
\end{align*}
At $t=0$ (just after the investment switch), this simplifies to:
\[
\frac{c_{m}}{\hat{q}_{1}}+c_{h}<\frac{c_{m}}{\hat{q}_{1}^{2}},
\]
or equivalently:
\begin{align*}
\frac{c_{m}}{c_{h}}\left(\frac{1}{\hat{q}_{1}^{2}}-\frac{1}{\hat{q}_{1}}\right) & >1,\\
\frac{c_{m}}{c_{h}}\frac{1-\hat{q}_{1}}{\hat{q}_{1}^{2}} & >1,\\
\frac{c_{m}}{c_{h}} & >\frac{\hat{q}_{1}^{2}}{1-\hat{q}_{1}},\\
\frac{c_{m}}{c_{h}} & >Q(\hat{q}_{1}).
\end{align*}
Thus, $\frac{c_{m}}{c_{h}}>Q(\hat{q}_{1})$ is a sufficient condition
to ensure $\frac{\partial\hat{T}}{\partial\rho}<0$. Intuitively,
if machine costs dominate human costs, increasing the discount rate
makes the switch occur earlier ($\hat{T}\downarrow$).
\end{proof}
\begin{cor}
The machine-to-human comparative advantage bound decreases as the
initial AI capability of the disadvantaged task increases.
\end{cor}
\begin{proof}
Using Lemma 2, we have:
\[
Q(\hat{q}_{1})=\frac{\hat{q}_{1}^{2}}{1-\hat{q}_{1}}.
\]
Taking the derivative with respect to $q_{2,0}$:
\[
\frac{\partial Q(\hat{q}_{1})}{\partial q_{2,0}}=\frac{\partial\left(\frac{\hat{q}_{1}^{2}}{1-\hat{q}_{1}}\right)}{\partial\hat{q}_{1}}\frac{\partial\hat{q}_{1}}{\partial q_{2,0}}<0.
\]
This implies that as $q_{2,0}$ increases, maintaining $\frac{\partial\hat{T}}{\partial\rho}<0$
requires less comparative machine cost advantage (i.e., lower $\frac{c_{m}}{c_{h}}$).
\end{proof}

\end{document}
