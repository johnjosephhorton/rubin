%% LyX 2.4.2.1 created this file.  For more info, see https://www.lyx.org/.
%% Do not edit unless you really know what you are doing.
\documentclass[english]{article}
\usepackage[T1]{fontenc}
\usepackage[latin9]{inputenc}
\usepackage{babel}
\usepackage{float}
\usepackage{amsmath}
\usepackage[a4paper]{geometry}
\geometry{verbose,tmargin=2cm,bmargin=2cm,lmargin=2cm,rmargin=2cm,footskip=0.75cm}
\PassOptionsToPackage{normalem}{ulem}
\usepackage{ulem}
\usepackage[]
 {hyperref}
\begin{document}
\setcounter{page}{0}
\title{General Equilibrium Sketch}

\maketitle

\pagenumbering{arabic}

I provide a sketch of the version we had in the previous draft with
some modifications. At the end, I give one or two ideas about other
ways we can think about modifying and extending the model. We can
discuss further on Monday.

\section{Task-based Production Environment}

We have a task-based model of production, with the following characteristics:
\begin{itemize}
\item Production of a consumption good $X$ requires completing an ordered
set of tasks, denoted by:
\[
\mathcal{T}=\{1,2,...,n\}.
\]
We refer to $\mathcal{T}$ as the ``production process'' of the
good. Later in the GE section where we have more than a single good
we use subscript $k$ for all variables introduced in this section,
but for now we are focusing on only a single good--so no subscripts.
\item Each task can be completed either manually or using an AI. If done
by an AI, the task is part of an 
\item Each task $i\in\mathcal{T}$ has three cost parameters: skill (i.e.,
human capital), time (i.e., labor), and hand-off (or context-switching)
cost. Denote the (skill, time, hand-off time) costs of task $i$ with
$(c_{i},t_{i},t_{i}^{s})$. The skill cost $c_{i}$ is a one-off cost
that is paid by the firm to train the worker to do task $i$. Once
trained, the firm employs the worker for the duration of task's time
cost $t_{i}$ to produce of one unit of the good.\\
\\
Hand-off cost $t_{i}^{s}$ is a frictional cost that a worker must
pay to hand-off output of his/her task to the next worker. Hand-off
can be viewed as an additional task at the end of task $i$ that takes
time $t_{i}^{s}$ to complete but requires no additional skills. We
discuss hand-off costs a bit more after introducing jobs next.
\item Firms assign tasks to workers in ``jobs.'' A job is a contiguous
set of tasks in the production process $\mathcal{T}$. Jobs are mutually
exclusive and collectively cover all tasks in $\mathcal{T}$. Let
$\mathcal{J}_{j}$ denote the set of tasks associated with job $j$
assigned to worker $j$ (each worker is assigned to one job, so we
can use the same index for the worker and their job). Formally, assuming
the firm splits the production process into $J$ distinct jobs we
have:
\begin{align*}
\forall j\neq j^{'},\quad j,j^{'}\in\{1,2,...,J\}: & \qquad\mathcal{J}_{j}\cap\mathcal{J}_{j^{'}}=\emptyset,\\
\end{align*}
and
\[
\bigcup_{j=1}^{J}\mathcal{J}_{j}=\mathcal{T}.
\]
\item Hand-off cost of a task is only realized when the task falls at the
boundary of two consecutive jobs $j$ and $j+1$. That is, if task
$i$ is the last task in job $j$, work $j$ incurs an additional
time cost $t_{i}^{s}$ to hand-off the output to the next worker.
Every task other than $i$ in job $j$ has zero realized hand-off
time cost. With a slight abuse of notation, we show hand-off cost
of job $j$ with $t_{j}^{s}:=t_{i}^{s}$. \\
\\
Remember that even though we are using notation $t_{j}^{s}$ to show
the hand-off time cost of job $j$ it is an inherent property of the
\uline{last task of the job}, not the job itself. So, $t_{j}^{s}$
is actually determined by the task at the boundary of job $j$, and
may differ from one job structure to another.
\item Continuing on the previous point, hand-off costs and thus production
costs depend on structure of jobs or ``job design.'' But how are
costs determined?\\
Consider some collection of jobs $\mathcal{J}=\{\mathcal{J}_{j}\}_{j=1}^{J}$
containing $J$ distinct jobs. The wage that has to be paid to worker
employed in job $j$ is given by
\begin{equation}
w_{j}:=v_{L}+\sum_{i\in\mathcal{J}_{j}}c_{i},\label{eq:wage}
\end{equation}
where $v_{L}$ is worker's value of leisure.\footnote{To make workers indifferent between working at different jobs the
firm compensates them for the job's required skill costs. Moreover,
the firm pays workers the value of their leisure to make them indifferent
between working and not working.} The time (or labor) cost of job $j$ is:
\begin{equation}
t_{j}:=t_{j}^{s}+\sum_{i\in\mathcal{J}_{j}}t_{i}.\label{eq:time}
\end{equation}
\item The wage bill for job $j$ is the product of $w_{j}$ and $t_{j}$:
\begin{equation}
WageBill_{j}=w_{j}t_{j},\label{eq:wage_bill}
\end{equation}
which represents workers being compensated proportionally to their
skill level and the time required to complete the job. Producing one
unit of the consumption good $X$ only involves paying the labor time
cost of workers performing tasks directly associated with that unit,
as the one-time human capital training costs have already been incurred.
\item With this production function, optimal allocation of tasks to jobs
(and thus job boundaries) are determined by solving firm's cost minimization
problem:
\[
\min_{J,\mathcal{J}=\{\mathcal{J}_{j}\}_{j=1}^{J}}\ \ \sum_{j=1}^{J}c_{j}t_{j}=\sum_{j=1}^{J}\left[\left(v_{L}+\sum_{i\in\mathcal{J}_{j}}c_{i}\right)\left(t_{j}^{s}+\sum_{i\in\mathcal{J}_{j}}t_{i}\right)\right].
\]
(We have a section dedicated to solving this problem via dynamic programming.)
\item To produce one unit of the consumption good $X$, each task in the
production process must be performed exactly once: it cannot be the
case that, for example, task 1 is done twice while task 2 is done
once because task 1 is cheaper. All tasks must be done in fixed proportions.
Thus, at the level of tasks, production function takes a Leontief
form.
\item Completion of a task creates an ``intermediate good,'' which is
used as input in the production of the final good. Each worker produces
intermediate goods by performing tasks assigned to them in their job.
The firm buys the intermediate goods (i.e., output of tasks) from
workers in a perfectly competitive market, and produces the good at
no additional cost. \\
\\
Without loss, we can imagine each worker produces only a single intermediate
good--the output of the last task assigned to them in their job (think
of it as output of each task being the input of the next task; The
worker produces intermediate task outputs and uses them as inputs
to ultimately produce the output of the last task assigned to them).
Let $x_{j}$ show the intermediate good produced in job $j$. The
production function of the consumption good $X$ can be written as:
\[
X(x_{1},...,x_{J})=min(x_{1},...,x_{J}).
\]
\end{itemize}

\section{General Equilibrium}

This section is structured as follows: I introduce the $K$-goods
production economy first. Then describe household's preferences and
market clearing conditions to ultimately get the equilibrium allocations
and prices. Before introducing market clearing, I discuss how prices
are determined as a function of model primitives as it helps the discussion
in that subsection.

\subsection{Production}

The AI's success probability in completing task $i$, denoted by $q_{i}$,
in every attempt is
\[
q_{i}=\alpha^{d_{i}},
\]
where $\alpha\in(0,1)$ is the (general) AI quality and $d_{i}>0$
is the task's completion difficulty. The AI's (expected) time cost
to complete task $i$ is:
\[
\frac{t_{i}^{m}}{q_{i}}=\frac{t_{i}^{m}}{\alpha^{d_{i}}}=t_{i}^{m}\alpha^{-d_{i}},
\]
where $t_{i}^{m}$ is the management cost of task $i$. In contrast,
the time cost of completing task $i$ manually by human is $t_{i}^{h}$.

Each task is either done manually by a human or is a part of an AI-completed
chain. If the task $i$ is done manually, its time cost is simply
just $t_{i}^{h}$. If task $i$ is a part of an AI chain $c$ it is
either augmented or automated. As discussed earlier in the paper,
the success probability of task $i$ appears as a multiplicative term
in the success probability of the chain $q_{c}$. If task $i$ is
augmented, it also determines the management time of the chain via
$t_{i}^{m}$ whereas in the case of automation its contribution to
the chain cost remains at the success probability alone.

In any job design, each job is (potentially) comprised by tasks that
are done by humans manually and/or AI chains. Fix some job design
$\mathcal{J}^{k}=\{\mathcal{J}_{j}\}_{j=1}^{J_{k}}$. Let us show
the set of tasks in job $j$ that are done by human with $\mathcal{H}_{j}^{k}(\mathcal{J}^{k})$
and the set of AI chains in job $j$ that are done by the AI with
$\mathcal{C}_{j}^{k}(\mathcal{J}^{k})$. Let us show the cost of a
chain $c\in\mathcal{C}_{j}$ with $\frac{t_{c}}{q_{c}}$. In this
notation, $t_{c}$ is the management time of the augmented task in
the chain (for example, if task $i\in c$ is the augmented task, then
$t_{c}=t_{i}^{m}$) and $q_{c}$ is the composite success probability
which is defined as $q_{c}=\prod_{i\in c}q_{i}$.

The production function at the level of jobs, which produces intermediate
good $x_{j}$ can be characterized as:
\[
x_{j}^{k}=f(l^{hk},l^{mk};\alpha)=\min\left(\min_{i\in\mathcal{H}_{j}^{k}}\frac{l_{i}^{h}}{t_{i}^{h}},\min_{c\in\mathcal{C}_{j}^{k}}\frac{l_{c}^{m}}{t_{c}^{m}\alpha^{-d_{c}}}\right),
\]
where $l_{i}^{h}$ is the labor dedicated to manual task $i$, $l_{c}^{m}$
is the labor allocated to augment the last task in AI chain $c$. 

The production function of consumption good $k$ takes the following
form:

\begin{equation}
X_{k}=F(l_{1}^{hk},\cdots,l_{J_{k}}^{hk},l_{1}^{mk},\cdots,l_{J_{k}}^{mk})=min(x_{1}^{k},\cdots,x_{J_{k}}^{k}),\label{eq:cons_prodfun}
\end{equation}
reflecting that one unit of each intermediate good is required to
produce one unit of the consumption good.

\subsection{Households}

A representative household supplies one unit of labor inelastically
and has a CES utility function over the $K$ goods:
\[
U(X_{1},...,X_{K})=\left(\sum_{k=1}^{K}\delta_{k}X_{k}^{\frac{\sigma-1}{\sigma}}\right)^{\frac{\sigma}{\sigma-1}},\qquad0<\delta_{k}<1,\qquad\sum_{k=1}^{K}\delta_{k}=1,\qquad\sigma>0.
\]
Goods are substitutes if $\sigma>1$, complements if $\sigma<1$,
and $\sigma=1$ corresponds to the Cobb--Douglas case. The household
faces the budget constraint
\begin{equation}
\sum_{k=1}^{K}p^{k}X_{k}=I,\label{eq:budget_constraint}
\end{equation}
where $I$ is the total income earned from supplying labor to jobs
across all sectors. The income $I$ is the total sum of earnings from
producing intermediate goods, which the household takes as given when
choosing how much of each good to consume.

\subsection{Goods Prices}

\subsubsection{Intermediate Goods Prices}

Per unit of intermediate good $x_{j}^{k}$, the firm employs $l_{i}^{h}$
amount of labor on manual tasks $i\in\mathcal{H}_{j}^{k}$ and $l_{c}^{m}$
amount of labor on each AI-completed chain $c\in\mathcal{C}_{j}^{k}$.
That is:
\begin{align*}
\forall i\in\mathcal{H}_{j}^{k} & :l_{i}^{h}=t_{i}^{h},\\
\forall c\in\mathcal{C}_{j}^{k} & :l_{c}^{m}=t_{c}^{m}\alpha^{-d_{c}}.
\end{align*}
Moreover, define the total amount of labor spent on production of
intermediate good $x_{j}^{k}$ as:
\begin{align*}
l_{j}^{k} & :=t_{j}^{s}+\sum_{i\in\mathcal{H}_{j}^{k}}l_{i}^{h}+\sum_{c\in\mathcal{C}_{j}^{k}}l_{c}^{m}\\
 & =t_{j}^{s}+\sum_{i\in\mathcal{H}_{j}^{k}}t_{i}^{h}+\sum_{c\in\mathcal{C}_{j}^{k}}t_{c}^{m}\alpha^{-d_{c}}
\end{align*}
The per unit production cost (i.e., the wage bill) of good $x_{j}^{k}$
can be rewritten as:
\begin{align*}
WageBill_{j} & =w_{j}^{k}l_{j}^{k}\\
 & =w_{j}^{k}\left(t_{j}^{s}+\sum_{i\in\mathcal{H}_{j}^{k}}t_{i}^{h}+\sum_{c\in\mathcal{C}_{j}^{k}}t_{c}^{m}\alpha^{-d_{c}}\right)
\end{align*}
This just a rephrasing of equation (\ref{eq:wage_bill}). \\
\\
Markets are perfectly competitive, therefore the price of intermediate
good $x_{j}^{k}$, denoted by $p_{j}^{k}$, equals the marginal cost
of its production:
\begin{align}
p_{j}^{k} & =w_{j}^{k}l_{j}^{k}\label{eq:intermediate_good_price}\\
 & =w_{j}^{k}\left(t_{j}^{s}+\sum_{i\in\mathcal{H}_{j}^{k}}t_{i}^{h}+\sum_{c\in\mathcal{C}_{j}^{k}}t_{c}^{m}\alpha^{-d_{c}}\right)
\end{align}


\subsubsection{Consumption Goods Prices}

Given the Leontief nature of production at the job level, perfect
competition in the consumption goods market implies that price of
$X_{K}$, $p^{k}$, equals the sum of the prices of its constituent
intermediate goods:
\begin{align*}
p^{k} & =\sum_{j=1}^{J_{k}}p_{j}\\
 & =\sum_{j=1}^{J_{k}}w_{j}^{k}l_{j}^{k}\\
 & =\sum_{j=1}^{J_{k}}w_{j}^{k}\left(t_{j}^{s}+\sum_{i\in\mathcal{H}_{j}^{k}}t_{i}^{h}+\sum_{c\in\mathcal{C}_{j}^{k}}t_{c}^{m}\alpha^{-d_{c}}\right)
\end{align*}
To maintain consistency in notation between the final and intermediate
goods markets, define
\begin{align}
w^{k}l^{k} & \equiv\sum_{j=1}^{J_{k}}w_{j}^{k}l_{j}^{k}\label{eq:aggregate_price}\\
 & =\sum_{j=1}^{J_{k}}w_{j}^{k}\left(t_{j}^{s}+\sum_{i\in\mathcal{H}_{j}^{k}}t_{i}^{h}+\sum_{c\in\mathcal{C}_{j}^{k}}t_{c}^{m}\alpha^{-d_{c}}\right),
\end{align}
so that $p^{k}=w^{k}l^{k}$. \\
\\
Equation (\ref{eq:aggregate_price}) seems quite similar to equation
(\ref{eq:intermediate_good_price}). If we let $l^{k}$ be the total
amount of time (or labor) spent on producing good $X_{k}$, i.e.,
\begin{align*}
l^{k} & =\sum_{j=1}^{J_{k}}l_{j}^{k}\\
 & =\sum_{j=1}^{J_{k}}\left(t_{j}^{s}+\sum_{i\in\mathcal{H}_{j}^{k}}t_{i}^{h}+\sum_{c\in\mathcal{C}_{j}^{k}}t_{c}^{m}\alpha^{-d_{c}}\right),
\end{align*}
then $w^{k}$ can be interpreted as the weighted average wage rate
of the entire sector:
\begin{align*}
w^{k} & =\frac{\sum_{j=1}^{J_{k}}w_{j}^{k}l_{j}^{k}}{\sum_{j=1}^{J_{k}}l_{j}^{k}}\\
 & =\frac{\sum_{j=1}^{J_{k}}w_{j}^{k}\left(t_{j}^{s}+\sum_{i\in\mathcal{H}_{j}^{k}}t_{i}^{h}+\sum_{c\in\mathcal{C}_{j}^{k}}t_{c}^{m}\alpha^{-d_{c}}\right)}{\sum_{j=1}^{J_{k}}\left(t_{j}^{s}+\sum_{i\in\mathcal{H}_{j}^{k}}t_{i}^{h}+\sum_{c\in\mathcal{C}_{j}^{k}}t_{c}^{m}\alpha^{-d_{c}}\right)}.
\end{align*}

Finally, note that once $l^{k}$ is determined, all $l_{j}^{k}$s
are automatically pinned down. This is due to the assumption made
in the second to last bullet point of Section 1 that tasks are done
in fixed proportions. So, it is without loss to only focus on working
with $l^{k}$ and ignore $l_{j}^{k}$s for the rest of the analysis.

\subsection{Labor Market Clearing}

The (representative) household supplies one unit of labor inelastically
to all jobs across all sectors. Since $l$ is the per unit of good
labor employed the labor market clearing condition satisfies:
\begin{equation}
\sum_{k=1}^{K}\left(\sum_{j=1}^{J_{k}}l_{j}^{k}x_{j}^{k}\right)=1.\label{eq:labor_market_clearing}
\end{equation}
Given the Leontief production at task level, (\ref{eq:labor_market_clearing})
can be expressed as:
\begin{align}
\sum_{k=1}^{K}\left(\sum_{j=1}^{J_{k}}l_{j}^{k}x_{j}^{k}\right) & =\sum_{k=1}^{K}\left(\sum_{j=1}^{J_{k}}l_{j}^{k}X_{k}\right)\label{eq:labor_market_clearing_simplified}\\
 & =\sum_{k=1}^{K}l^{k}X_{k}\\
 & =1
\end{align}
With this, we can now derive household's earned labor income $I$
mentioned in (\ref{eq:budget_constraint}) in term of the primitives
of the problem:
\begin{align*}
I & =\sum_{k=1}^{K}p^{k}X_{k}\\
 & =\sum_{k=1}^{K}\left(\sum_{j=1}^{J_{k}}p_{j}^{k}\right)X_{k}\\
 & =\sum_{k=1}^{K}\left(\sum_{j=1}^{J_{k}}w_{j}^{k}l_{j}^{k}\right)X_{k}\\
 & =\sum_{k=1}^{K}\left(\sum_{j=1}^{J_{k}}\left(v_{L}+\sum_{i\in\mathcal{H}_{j}^{k}}c_{i}^{h}+\sum_{c\in\mathcal{C}_{j}^{k}}c_{c}^{m}\right)l_{j}^{k}\right)X_{k}\\
 & =v_{L}\underbrace{\sum_{k=1}^{K}\left(\sum_{j=1}^{J_{k}}l_{j}^{k}\right)X_{k}}_{=1}+\sum_{k=1}^{K}\left(\sum_{j=1}^{J_{k}}\left(\sum_{i\in\mathcal{H}_{j}^{k}}c_{i}^{h}+\sum_{c\in\mathcal{C}_{j}^{k}}c_{c}^{m}\right)l_{j}^{k}\right)X_{k}\\
 & =v_{L}+\sum_{k=1}^{K}\left(\sum_{j=1}^{J_{k}}\left(\sum_{i\in\mathcal{H}_{j}^{k}}c_{i}^{h}+\sum_{c\in\mathcal{C}_{j}^{k}}c_{c}^{m}\right)\left(t_{j}^{s}+\sum_{i\in\mathcal{H}_{j}^{k}}t_{i}^{h}+\sum_{c\in\mathcal{C}_{j}^{k}}t_{c}^{m}\alpha^{-d_{c}}\right)\right)X_{k},
\end{align*}
where we used (\ref{eq:wage}) to arrive at the last line. The household's
budget thus consists of two parts: a part that compensates the worker
for their forgone leisure and a part that rewards them proportional
to how much they produce consumption goods.

\subsection{Equilibrium}

Here we solve for the equilibrium allocations and sectoral employment.
The consumer chooses $X_{k}$ to maximize
\[
U(x_{1},\dots,x_{K})=\Bigl(\sum_{k=1}^{K}\delta_{k}\,x_{k}^{\tfrac{\sigma-1}{\sigma}}\Bigr)^{\tfrac{\sigma}{\sigma-1}},
\]
subject to
\[
\sum_{k=1}^{K}p^{k}X_{k}=I.
\]
By the standard first-order conditions, the ratio of marginal utilities
equals the ratio of prices for any two goods $f$ and $g$:
\[
\frac{\partial U/\partial X_{f}}{\partial U/\partial X_{g}}=\frac{p^{f}}{p^{g}}.
\]
From CES properties, we obtain:
\[
\left(\frac{X_{f}}{X_{g}}\right)^{-\tfrac{1}{\sigma}}=\frac{p^{f}}{p^{g}}\,\cdot\,\frac{\delta_{g}}{\delta_{f}},\quad\Longrightarrow\quad\frac{X_{f}}{X_{g}}=\left(\frac{p^{g}}{p^{f}}\,\cdot\,\frac{\delta_{f}}{\delta_{g}}\right)^{\sigma}.
\]
Define 
\[
R_{f}=\frac{X_{f}}{X_{1}}=\Bigl(\frac{p^{1}}{p^{f}}\cdot\frac{\delta_{f}}{\delta_{1}}\Bigr)^{\sigma},
\]
with $R_{1}=1$. Substitute $X_{f}=R_{f}\,X_{1}$ into the simplified
labor market clearing condition (\ref{eq:labor_market_clearing_simplified}):
\[
\sum_{k=1}^{K}l^{k}\,(R_{k}\,X_{1})=x_{1}\Bigl[\sum_{k=1}^{K}l^{k}\,R_{k}\Bigr]=1,
\]
which yields 
\[
X_{1}^{*}=\frac{1}{\sum_{k=1}^{K}l^{k}\,R_{k}},\quad X_{f}^{*}=R_{f}\,X_{1}^{*}=\frac{R_{f}}{\sum_{k=1}^{K}l^{k}\,R_{k}}.
\]

Amount of labor employed in sector $k$ is given by $L_{k}^{*}=l^{k}\,X_{k}^{*}$.
Thus, the equilibrium can be summarized as:
\begin{align*}
w_{j}^{k}=\Biggl(v_{L}+\sum_{i\in\mathcal{J}_{j}^{k}}c_{i}^{k}\Biggr), & \qquad l_{j}^{k}=\Biggl(t_{j}^{sk}+\sum_{i\in\mathcal{J}_{j}^{k}}t_{i}^{k}\Biggr),\\
p_{j}^{k}=w_{j}^{k}l_{j}^{k}, & \qquad p^{k}=w^{k}l{}^{k}=\sum_{j=1}^{J_{k}}w_{j}^{k}l_{j}^{k},\\
X_{k}^{*}=\frac{R_{k}}{\sum_{\ell=1}^{K}l^{\ell}\,R_{\ell}}, & \qquad L_{k}^{*}=l^{k}\,X_{k}^{*}=\sum_{j=1}^{J_{k}}l_{j}^{k}\,X_{k}^{*},
\end{align*}
where
\[
R_{k}=\Bigl(\frac{p^{1}}{p^{k}}\cdot\frac{\delta_{k}}{\delta_{1}}\Bigr)^{\sigma}.
\]
Rewriting these as a function of primitives we have: 
\begin{align*}
w_{j}^{k}=\left(v_{L}+\sum_{i\in\mathcal{H}_{j}^{k}}c_{i}^{h}+\sum_{c\in\mathcal{C}_{j}^{k}}c_{c}^{m}\right), & \qquad l_{j}^{k}=\left(t_{j}^{s}+\sum_{i\in\mathcal{H}_{j}^{k}}t_{i}^{h}+\sum_{c\in\mathcal{C}_{j}^{k}}t_{c}^{m}\alpha^{-d_{c}}\right),\\
p_{j}^{k}=w_{j}^{k}l_{j}^{k}=\left(v_{L}+\sum_{i\in\mathcal{H}_{j}^{k}}c_{i}^{h}+\sum_{c\in\mathcal{C}_{j}^{k}}c_{c}^{m}\right)\left(t_{j}^{s}+\sum_{i\in\mathcal{H}_{j}^{k}}t_{i}^{h}+\sum_{c\in\mathcal{C}_{j}^{k}}t_{c}^{m}\alpha^{-d_{c}}\right), & \qquad p^{k}=w^{k}l{}^{k}=\sum_{j=1}^{J_{k}}w_{j}^{k}l_{j}^{k},\\
X_{k}^{*}=\frac{R_{k}}{\sum_{\ell=1}^{K}\left[\sum_{j=1}^{J_{\ell}}\left(t_{j}^{s}+\sum_{i\in\mathcal{H}_{j}^{\ell}}t_{i}^{h}+\sum_{c\in\mathcal{C}_{j}^{\ell}}t_{c}^{m}\alpha^{-d_{c}}\right)\right]R_{\ell}}, & \qquad L_{k}^{*}=l^{k}\,X_{k}^{*}=\left[\sum_{j=1}^{J_{k}}\left(t_{j}^{s}+\sum_{i\in\mathcal{H}_{j}^{k}}t_{i}^{h}+\sum_{c\in\mathcal{C}_{j}^{k}}t_{c}^{m}\alpha^{-d_{c}}\right)\right]X_{k}^{*},
\end{align*}
In this way, the job design of the production process $\mathcal{T}_{k}$
(and thus the corresponding skill and time costs of tasks) determines
goods prices $p^{k}$, which then determine the consumption bundle
$\{X_{k}^{*}\}_{k=1}^{K}$ and the sectoral labor allocations $\{L_{k}^{*}\}_{k=1}^{K}$.
Notice that the equilibrium outcome also depends implicitly on job
designs through $J_{k}s$ and $t_{j}^{sk}$s. 

\section{Thoughts, Comments, Ideas}

Below I give some thoughts on the model. The points are structured
into three items: a reframing proposal, an observation about current
modeling assumptions, and a concern regarding integration with existing
literature.
\begin{enumerate}
\item \textbf{Reframing how consumers enter the model}: This first point
suggests a minor restructuring of how consumers enter the market.
This proposal is essentially a reframing of the earlier version and
does not significantly alter the core outcomes, but I would still
appreciate discussing it further in person. Currently, consumption
goods are produced directly through a labor-only production function,
placing household preferences as the primary source of economic variation.
I'm unsure if this framing is standard in the literature. Alternatively,
we might reframe these consumption goods as intermediate inputs used
in a subsequent CES production function, which produces a single final
good directly sold to households. Although the mathematical implications
would remain identical, the conceptual shift could better align our
model with standard practices.
\item \textbf{Observation on job design integration into GE}: Currently,
I implicitly treat job designs as exogenously given, a condition that
doesn't naturally emerge endogenously from the GE setup. While firm's
cost-minimization problem could be interpreted as an endogenous decision
itself, I remain somewhat unhappy with the existing separation of
these two aspects. I would appreciate feedback or ideas on how we
might integrate the cost-minimization decision more naturally within
the GE framework.
\item \textbf{Concern about connection to literature}: Our current model
notably diverges from common task-based models in the literature.
Ideally, we would want to link our framework somehow to the task model
of \href{https://economics.mit.edu/sites/default/files/publications/Skills\%2C\%20Tasks\%20and\%20Technologies\%20-\%20Implications\%20for\%20.pdf}{Acemoglu and Autor (2011)},
which is pretty well-known and accepted. However, I don't quite know
how we can get this connection, primarily because of our unique concept
of ``chaining,'' where an individual task might be more cost-efficient
when performed by humans but more efficient overall when automated
as part of a sequence of tasks (we currently have a proposition on
this in the paper).\\
\\
Directly adopting AA's approach into our current framework would entail
representing task-level choices between human and machine labor for
instance like: (see p. 1121 of handbook chapter for reference)
\[
x(i)=A_{human}\alpha_{human}(i)h(i)+A_{machine}\alpha_{machine}(i)m(i).
\]
Here, the $\alpha$s represent productivity parameters for each task
$i$, and $h(i)$ and $m(i)$ denote the two types of workers (human
or machine). Initially, I avoided this formulation because explicitly
incorporating our ``chaining'' concept seemed challenging at first
glance (if not impossible). Incorporating the chaining nuances explicitly
into this more generalized task-based framework would likely require
substantial restructuring, and I remain skeptical about its feasibility
without heavy modifications.
\end{enumerate}

\end{document}
