%% Cleaned version
\documentclass[english]{article}
\usepackage[T1]{fontenc}
\usepackage[latin9]{inputenc}
\usepackage{babel}
\usepackage[a4paper]{geometry}
\usepackage{array}
\geometry{verbose,tmargin=2cm,bmargin=2cm,lmargin=2cm,rmargin=2cm,footskip=0.75cm}

% Math and formatting
\usepackage{amsmath,amssymb,amsfonts,amsthm,mathtools}
\usepackage{xcolor}
\usepackage{float}
\usepackage{esint}
\PassOptionsToPackage{normalem}{ulem}
\usepackage{ulem}
\usepackage{comment}
\usepackage[colorlinks=true,linkcolor=blue,citecolor=blue,urlcolor=blue]{hyperref}
\usepackage[nameinlink,noabbrev,capitalise]{cleveref}

% Environments
\newtheorem{proposition}{Proposition}
\newtheorem{corollary}{Corollary}
\newtheorem{remark}{Remark}
\newtheorem{definition}{Definition}
\newtheorem{proofpart}{Part}

\title{General Equilibrium and Comparative Statics}
\author{}
\date{}

\begin{document}
\maketitle
\pagenumbering{arabic}


\begin{comment}

\section*{Reference: Equations (8)--(19) from the paper}
These are equations copy-pasted from the paper:

\begin{gather}
x = \min\!\left\{\frac{\bar{\alpha}\, l_M}{\tau_M},\ \frac{l_H}{\tau_H}\right\} \tag{8}\label{eq:8-orig}
\\[0.5em]
\tau_M = \sum_{b=1}^{k} \tau^{M}_{b} \tag{9}\label{eq:9-orig}
\\[0.5em]
\tau_H = \sum_{j=1}^{p-1} \tau_{S}(J_j) + \sum_{b=k+1}^{n} \tau^{H}_{b} \tag{10}\label{eq:10-orig}
\\[0.5em]
\bar{\alpha} = \frac{\sum_{b=1}^{k} \tau^{M}_{b}}{\sum_{b=1}^{k} \tau^{M}_{b}\,\alpha^{-d_b}} \tag{11}\label{eq:11-orig}
\\[0.5em]
x = \frac{\bar{\alpha}\, l_M}{\tau_M} = \frac{l_H}{\tau_H} \tag{12}\label{eq:12-orig}
\\[0.5em]
X = \big(\theta_M M^{\rho} + \theta_H H^{\rho} + (1-\theta_M-\theta_H)K^{\rho}\big)^{1/\rho} \tag{13}\label{eq:13-orig}
\\[0.5em]
\frac{w_M \tau_M}{1 - w_H \tau_H} \le \bar{\alpha} \le 1 \tag{14}\label{eq:14-orig}
\\[0.5em]
X = \int_{\,\frac{w_M \tau_M}{1-w_H \tau_H}}^{1} \phi(\bar{\alpha})\,d\bar{\alpha} \tag{15}\label{eq:15-orig}
\\[0.5em]
M = \int_{\,\frac{w_M \tau_M}{1-w_H \tau_H}}^{1} \tau_M \frac{\phi(\bar{\alpha})}{\bar{\alpha}}\,d\bar{\alpha} \tag{16}\label{eq:16-orig}
\\[0.5em]
H = \int_{\,\frac{w_M \tau_M}{1-w_H \tau_H}}^{1} \tau_H \phi(\bar{\alpha})\,d\bar{\alpha} \tag{17}\label{eq:17-orig}
\\[0.5em]
\int_{\,\frac{w_M \tau_M}{1-w_H \tau_H}}^{1} \phi(\bar{\alpha})\,d\bar{\alpha}
= \Bigg(
\theta_M \Big[\tau_M \!\!\int_{\,\frac{w_M \tau_M}{1-w_H \tau_H}}^{1} \frac{\phi(\bar{\alpha})}{\bar{\alpha}}\,d\bar{\alpha}\Big]^{\rho}
+ \theta_H \Big[\tau_H \!\!\int_{\,\frac{w_M \tau_M}{1-w_H \tau_H}}^{1} \phi(\bar{\alpha})\,d\bar{\alpha}\Big]^{\rho}
+ (1-\theta_M-\theta_H)
\Bigg)^{\!1/\rho} \tag{18}\label{eq:18-orig}
\\[0.75em]
\phi(\bar{\alpha})
= \frac{\left(1-\theta_{M}-\theta_{H}\right)^{\frac{1}{\rho}}\left(1-\theta_{H}\,\tau_{H}^{\rho}\right)^{\frac{\rho}{\rho-1}}\left(\theta_{M}\,\tau_{M}^{\rho}\right)^{\frac{1}{1-\rho}}}{\rho-1}
(\bar{\alpha})^{\frac{2\rho-1}{1-\rho}}
\left[1-\theta_{H}\,\tau_{H}^{\rho}
- \left(1-\theta_{H}\,\tau_{H}^{\rho}\right)^{\frac{\rho}{\rho-1}}\left(\theta_{M}\,\tau_{M}^{\rho}\right)^{\frac{1}{1-\rho}}(\bar{\alpha})^{\frac{\rho}{1-\rho}}\right]^{-\frac{1+\rho}{\rho}} \tag{19} \label{eq:19-orig}
\end{gather}

\end{comment}


\bigskip

\section*{Setup}

\paragraph{Environment.}
\textcolor{red}{Add micro productions and $\phi$ to show how the macro function is obtained.}

Assume $\rho<0$, $\tau_H,\tau_M,\beta_H,\beta_M>0$, and normalize $K\equiv1$.
Aggregate output is:
\begin{align}
X \;=\; \Big(\theta_M M^\rho+\theta_H H^\rho+1-\theta_M-\theta_H\Big)^{1/\rho}, 
\end{align}
\[
S:=X^\rho \;=\; \theta_M M^\rho+\theta_H H^\rho+1-\theta_M-\theta_H.
\]
Normalize price of output $p=1$.


\paragraph{Consumption and Labor Supply.}
A single representative consumer consumes the output and supplies both types of labor:
\[
U = X - \frac{\beta_M}{2}M^2 - \frac{\beta_H}{2}H^2,
\]
with budget constraint:
\[
w_M M + w_H H = X.
\]
%where the LHS is consumer's labor income and the RHS is his expenditures.

Consumer's FOCs yields labor supply equations:
\begin{equation}
\beta_M M \;=\; w_M,
\qquad
\beta_H H \;=\; w_H. \label{eq:labor_supply}
\end{equation}


\paragraph{Consumption and Labor Demand.}
Perfect competition in output market implies MP equates MC:
\begin{equation}
w_M \;=\; \theta_M M^{\rho-1} S^{\frac{1}{\rho}-1},
\qquad
w_H \;=\; \theta_H H^{\rho-1} S^{\frac{1}{\rho}-1}, \label{eq:labor_demand}
\end{equation}
which are labor demand equations.

\paragraph{Micro Side and Market Participation Cutoff.}
Recall that $\phi(\bar{\alpha})$ is the effective AI quality distribution on $(0,1]$.
The participation cutoff implied by individual firm's profitability condition is
\begin{equation}
u \;=\; \frac{w_M\tau_M}{\,1-w_H\tau_H\,}
\qquad\text{(feasible if \ }w_H < \frac{1}{\tau_H}\text{)}. \label{eq:u_def}
\end{equation}

Define the micro integrals:
\begin{equation}
\Gamma(u;\phi):=\int_{u}^{1}\phi(\bar{\alpha})\,d\bar{\alpha},
\qquad
\Psi(u;\phi):=\int_{u}^{1}\frac{\phi(\bar{\alpha})}{\bar{\alpha}}\,d\bar{\alpha}, \label{eq:micro_integrals}
\end{equation}
and micro-to-macro aggregated human labor and AI management labor intensities:
\[
h(u;\phi)=\tau_H\,\Gamma(u;\phi),
\qquad 
m(u;\phi)=\tau_M\,\Psi(u;\phi).
\]
Micro-to-macro aggregation requires:
\begin{equation}
X=\Gamma(u;\phi),\quad (\text{i.e.}\quad S=\Gamma(u;\phi)^{\rho}),
\qquad 
\frac{M}{H}=\frac{m(u;\phi)}{h(u;\phi)}=\frac{\tau_M\,\Psi(u;\phi)}{\tau_H\,\Gamma(u;\phi)}.
\label{eq:AggMix}
\end{equation}

\section*{Equilibrium}
Setting labor supply \eqref{eq:labor_supply} equal to labor demand \eqref{eq:labor_demand}:
\begin{equation}
\beta_M \;=\; \theta_M M^{\rho-2} S^{\frac{1}{\rho}-1}.
\qquad
\beta_H \;=\; \theta_H H^{\rho-2} S^{\frac{1}{\rho}-1}, \label{eq:labor_demand_supply}
\end{equation}
Using $S=\Gamma^\rho$
\begin{equation}
H
= \left[ \frac{\beta_H}{\theta_H}\,\Gamma^{\,1-\frac{1}{\rho}} \right]^{\frac{1}{\rho-2}},
\qquad
M
= \left[ \frac{\beta_M}{\theta_M}\,\Gamma^{\,1-\frac{1}{\rho}} \right]^{\frac{1}{\rho-2}}. \label{eq:ge_allocations}
\end{equation}

The two equations in \eqref{eq:ge_allocations} express $H$ and $M$ in terms of $\Gamma$. 
Notice that $\Gamma$ depends on $H$ and $M$ itself through the participation threshold $u$ which in indirect function of $H$ and $M$ through wages $w_H$ and $w_M$.
Therefore, the equilibrium allocation is determined via some fixed point relation.

Once $H$ and $M$ are determined, we can use the labor supply equations in \eqref{eq:labor_supply} to pin down wages.
This completes specifying the equilibrium allocations and prices.


\section*{Comparative Statics w.r.t. Effective AI Quality}

\textcolor{red}{can be improved}
We consider how a change in effective AI quality parameter affects the aggregate human and AI management labor as well as their shares in the aggregate CES production function.
The proper notion of comparative static analysis is to perturb the distribution of effective AI quality $\phi(\bar{\alpha})$ and see how it affects $M,\,H,\,\theta_M,\,\theta_H$.

We need some definitinos first. 

\paragraph{General AI Quality Improvement.}
%Let $\{\phi_\varepsilon\}_\varepsilon$ be any differentiable family.
When $\phi(\bar{\alpha})$ changes the effect on outcome variables depend on how the distribution shifts.
Therefore, we will work with the derivative of $\phi$. 
Define
\[
\delta\phi(\bar{\alpha}):= \frac{\partial \phi(\bar{\alpha})}{\partial \bar{\alpha}} \Big(= \frac{\partial \phi_\varepsilon(\bar{\alpha})}{\partial \varepsilon} \Big)
\]
and the linear functionals
\[
A:=\frac{1}{\Gamma}\int_u^{1}\delta\phi(\bar{\alpha})\,d\bar{\alpha},
\qquad
B:=\frac{1}{\Psi}\int_u^{1}\frac{\delta\phi(\bar{\alpha})}{\bar{\alpha}}\,d\bar{\alpha},
\]
with boundary intensities
\[
\lambda:=\frac{\phi(u)}{\Gamma},
\qquad 
\mu:=\frac{\phi(u)}{u\,\Psi}.
\]

\paragraph{Cutoff Movements.}
As the participation cutoff $u$ is itself a function of $H$ and $M$ any perturbation of $\phi$ inevitably moves the participation threshold as well.
Substituting for wages from \eqref{eq:labor_supply} into \eqref{eq:u_def} expresses $u$ in terms of $M$ and $H$:
\[
u=\dfrac{\beta_M\tau_M M}{\,1-\beta_H\tau_H H\,}.
\]
Taking log derivatives:
\[
d\ln u
= d\ln \tau_M + d\ln M \;+\; \frac{\beta_H\tau_H H}{\,1-\beta_H\tau_H H\,}\Big(d\ln H + d\ln \tau_H\Big),
\]
where
\begin{equation}
\kappa:=\frac{\beta_H\tau_H H}{1-\beta_H\tau_H H}=\frac{w_H\tau_H}{1-w_H\tau_H}. \label{eq:kappa}
\end{equation}
Thus:
\begin{equation}
\boxed{
du = u\Big(d\ln M + \kappa\,d\ln H + d\ln \tau_M + \kappa\,d\ln \tau_H\Big)}
\label{eq:du}
\end{equation}

\paragraph{Derivation of Micro Integrals with Endogenous Cutoff.}
Deriving \eqref{eq:micro_integrals} with the presumption that the cutoff $u$ is also moving gives:
\begin{equation}
\frac{d\Gamma}{\Gamma}=A-\lambda\,du,
\qquad
\frac{d\Psi}{\Psi}=B-\mu\,du.\label{eq:micro_int_variation}
\end{equation}

Totally differentiate the ratio $M/H$ given in \eqref{eq:AggMix} and use \eqref{eq:micro_int_variation},
\begin{align}
d\ln\!\Big(\frac{M}{H}\Big)
&= d\ln\!\Big(\frac{\tau_M}{\tau_H}\Big) + d\ln\Psi - d\ln\Gamma  \nonumber
\\ 
&= d\ln\!\Big(\frac{\tau_M}{\tau_H}\Big)
+ (B-A) - (\mu-\lambda)\,du.\label{eq:ln_MH}
\end{align}
Substituting for $du$ from \eqref{eq:du} into \eqref{eq:ln_MH} gives
\[
d\ln M - d\ln H
= (B-A) \;+\; d\ln\!\Big(\frac{\tau_M}{\tau_H}\Big) - (\mu-\lambda)\, u\Big(d\ln M + \kappa\,d\ln H + d\ln \tau_M + \kappa\,d\ln \tau_H\Big).
\]
Expand the last term on the RHS and move all $d\ln M$ and $d\ln H$ terms to the LHS to get:
\begin{equation}
\boxed{
\big[\,1 + (\mu-\lambda)u\,\big]\,d\ln M
+ \big[\, -1 + (\mu-\lambda)u\,\kappa\,\big]\,d\ln H
=
(B-A) + d\ln\!\Big(\frac{\tau_M}{\tau_H}\Big)
 - (\mu-\lambda)u\big(d\ln \tau_M + \kappa\,d\ln \tau_H\big)
}
\tag{$\mathrm{R}_{\mathrm{gen}}$}\label{eq:Rgen}
\end{equation}







\paragraph{Total Derivative of Aggregate CES Function.}
Starting from the CES aggregation condition
\[
S = \Gamma^\rho = \theta_M M^\rho + \theta_H H^\rho + 1 - \theta_M - \theta_H,
\]
take total differentials on both sides:
\[
dS
= \rho\,\Gamma^{\rho-1} d\Gamma
= \rho\,\Gamma^{\rho-1} \Gamma \left( \frac{d\Gamma}{\Gamma} \right)
= \rho\,S \left( \frac{d\Gamma}{\Gamma} \right),
\]
and
\[
dS
= \rho\,\theta_M M^{\rho-1} dM + \rho\,\theta_H H^{\rho-1} dH
+ M^\rho\,d\theta_M + H^\rho\,d\theta_H - d\theta_M - d\theta_H.
\]
Divide through by \(S\):
\[
\rho \frac{d\Gamma}{\Gamma}
= \rho\,\frac{\theta_M M^{\rho-1}}{S} dM + \rho\,\frac{\theta_H H^{\rho-1}}{S} dH
+ \frac{M^\rho - 1}{S} d\theta_M + \frac{H^\rho - 1}{S} d\theta_H.
\]

Now use \eqref{eq:labor_demand_supply} to eliminate
\(d\theta_M\) and \(d\theta_H\) in terms of \(d\ln M\), \(d\ln H\), and \(d\Gamma/\Gamma\).  
Specifically, from \eqref{eq:labor_demand_supply}:
\begin{equation}
d\ln\theta_M = -(\rho-2)\,d\ln M - (1-\rho)\,\frac{d\Gamma}{\Gamma},
\qquad
d\ln\theta_H = -(\rho-2)\,d\ln H - (1-\rho)\,\frac{d\Gamma}{\Gamma}. \label{eq:d_theta}
\end{equation}
Substituting into the expression for \(dS/S\) above yields:
\[
\left[ 2s_M + (\rho-2)\frac{\theta_M}{1-\theta_M-\theta_H}s_K \right] d\ln M
+ \left[ 2s_H + (\rho-2)\frac{\theta_H}{1-\theta_M-\theta_H}s_K \right] d\ln H
= \rho\frac{d\Gamma}{\Gamma} - (1-\rho)\frac{s_K}{1-\theta_M-\theta_H} \frac{d\Gamma}{\Gamma},
\]
where 
\begin{equation}
s_M = \frac{\theta_M M^\rho}{S}, \quad
s_H = \frac{\theta_H H^\rho}{S}, \quad 
s_K = \frac{1-\theta_M-\theta_H}{S}. \label{eq:s_def}
\end{equation}
Factor the common term in front of \(d\Gamma/\Gamma\) as
\begin{equation}
C := 1 - (1-\rho)\frac{s_K}{1-\theta_M-\theta_H}. \label{eq:C_def}
\end{equation}

Finally, substitute for $du$ from \eqref{eq:du} into \eqref{eq:micro_int_variation} and obtain $d\Gamma/\Gamma$ in terms of $u$:
\[
\frac{d\Gamma}{\Gamma}
= 
A - \lambda\, u \Big(d\ln M + \kappa\,d\ln H + d\ln \tau_M + \kappa\,d\ln \tau_H\Big)
\]
Plug this into the earlier equation and rearranging terms gives
\begin{equation}
\boxed{
\big[A_M'+C\,\lambda\,u\big]\,d\ln M \;+\; \big[A_H'+C\,\lambda\,u\,\kappa\big]\,d\ln H
= C\,A \;-\; C\,\lambda\,u\Big(d\ln \tau_M + \kappa\,d\ln \tau_H\Big)}
\tag{A}\label{eq:A}
\end{equation}
where
\begin{equation}
A_M' := 2s_M + (\rho-2)\frac{\theta_M}{1-\theta_M-\theta_H}s_K,
\qquad
A_H' := 2s_H + (\rho-2)\frac{\theta_H}{1-\theta_M-\theta_H}s_K. \label{eq:Aprime_def}
\end{equation}



\paragraph{Solving the $2\times 2$ Linear System.}
Equations \eqref{eq:Rgen} and \eqref{eq:A} form a coupled system in $d\ln M$ and $d\ln H$.
In matrix form:
\begin{equation}\label{eq:matrix-form}
\begin{bmatrix}
p_{11} & p_{12} \\
p_{21} & p_{22}
\end{bmatrix}
\begin{bmatrix}
d\ln M \\[2pt]
d\ln H
\end{bmatrix}
=
\begin{bmatrix}
q_{1} \\[2pt]
q_{2}
\end{bmatrix}.
\end{equation}
Here:
\[
p_{11} := 1+(\mu-\lambda)u, \quad p_{12} := -1+(\mu-\lambda)u\,\kappa,
\]
\[
p_{21} := A_M' + C\,\lambda\,u, \quad p_{22} := A_H' + C\,\lambda\,u\,\kappa,
\]
\[
q_{1}  := (B-A) + d\ln\!\Big(\tfrac{\tau_M}{\tau_H}\Big) 
          - (\mu-\lambda)u\Big(d\ln\tau_M+\kappa\,d\ln\tau_H\Big),
\]
\[
q_{2}  := C\,A - C\,\lambda\,u\Big(d\ln \tau_M + \kappa\,d\ln \tau_H\Big).
\]

\paragraph{Solution.}
Let
\[
\Delta := p_{11}p_{22} - p_{12}p_{21}
\]
Then
\begin{equation}
\boxed{~d\ln M=\frac{q_1 p_{22}-p_{12}q_2}{\Delta},\qquad
d\ln H=\frac{p_{11}q_2-q_1 p_{21}}{\Delta}.~}
\end{equation}

\paragraph{Wages, Cutoff, and CES Weights.}
Linear supply gives
\[
\boxed{~d\ln w_M=d\ln M,\qquad d\ln w_H=d\ln H.~}
\]
Cutoff:
\[
\boxed{~d\ln u=d\ln \tau_M + d\ln M \;+\; \kappa\big(d\ln H + d\ln \tau_H\big).~}
\]
And recall from \eqref{eq:d_theta}:
\begin{equation}
\boxed{~d\ln\theta_M=-(\rho-2)\,d\ln M-(1-\rho)\,\frac{d\Gamma}{\Gamma},\qquad
d\ln\theta_H=-(\rho-2)\,d\ln H-(1-\rho)\,\frac{d\Gamma}{\Gamma}.~}
\end{equation}



\section*{Implications}

Here we discuss the implications of the resutls.


\newpage


\begin{definition}[AI-tilt of $\phi$]
We say a $\phi$-shift $\delta\phi$ is \emph{AI-tilted} if it increases the \emph{AI-management labor intensity per unit of output} relative to human labor intensity at the given cutoff. Formally:
\[
B-A 
\;=\;   
\frac{1}{\Psi}\int_u^1\frac{\delta\phi}{\bar\alpha} \;-\; \frac{1}{\Gamma}\int_u^1\delta\phi 
\;>\; 
0.
\]
Equivalently, the proportional increase in $\Psi$ exceeds that in $\Gamma$---i.e., the distribution shifts weight toward high-$\bar\alpha$ firms in a way that raises the use of AI management labor ($\propto \int\phi/\bar\alpha$) more than human labor ($\propto \int\phi$).
\end{definition}



Notice that from \eqref{eq:ln_MH} and \eqref{eq:du} we have:
\[
\boxed{
d\ln\!\Big(\tfrac{M}{H}\Big)
=\underbrace{(B-A)}_{\text{$\phi$-tilt}}
\;+\;\underbrace{d\ln\tau_M-d\ln\tau_H}_{\text{time-cost gap}}
\;-\;\underbrace{(\mu-\lambda)\,u\Big(d\ln M+\kappa\,d\ln H+d\ln\tau_M+\kappa\,d\ln\tau_H\Big)}_{\text{cutoff feedback}},
}
\]
and from \eqref{eq:dtheta},
\[
\boxed{\ d\ln\!\Big(\tfrac{\theta_M}{\theta_H}\Big)=(2-\rho)\,d\ln\!\Big(\tfrac{M}{H}\Big)\ .\ }
\]

It is straightforward to see that $\frac{\theta_M}{\theta_H}$ moves in the same direction as $\frac{M}{H}$ as $2-\rho>0$ for $\rho<0$.



\textcolor{red}{Here is some sketch for a proposition:}
If we assume that $L = H + M$ so that we get a relationship between $dlnM$ and $dlnH$, namely:
\[
d\ln H = - \frac{M}{H} d\ln M
\]
We can substitute this into the expression for $dln(M/H)$ above. 
Assuming fixed automation strategy and job design so that $d\ln\tau_H =d\ln\tau_M=0$, we get:
\[
d\ln M + \frac{M}{H} d\ln M
=
(B-A)
\;-\;
(\mu-\lambda)\,u\Big(d\ln M-\kappa\,\frac{M}{H} d\ln M\Big),
\]
Rearranging gives:
\[
d\ln M 
= 
\frac{B - A}{1+\frac{M}{H}+(\mu-\lambda)u(1 - \kappa \frac{M}{H})}
=
\frac{B - A}{1+(\mu-\lambda)u + \frac{M}{H}(1 - \kappa (\mu-\lambda)u)}
\]
A sufficient condition for an AI-tilted shift to increase the ratio $M/H$ is:
\[
0 < \kappa (\mu-\lambda)u \leq 1
\]
or equivalently:
\[
\boxed{
0 < \mu-\lambda \leq \frac{1}{\kappa u}
}
\]

Implications of the derived condition:
\begin{enumerate}
    \item The lower bound $\mu - \lambda > 0$: This says that the marginal firm at the cutoff must be less AI labor-intensive than the average firm above the cutoff. This ensures that shifting the cutoff upward (e.g., by reallocating labor(?)) pushes the composition toward more AI-intensive firms, reinforcing the AI-tilt effect.
    \item The upper bound $\mu - \lambda \leq \frac{1}{\kappa u}$: This says that the composition effect(?) isn't too strong relative to labor market tightness(?). If $\mu - \lambda$ is too large, the induced movement from the cutoff could be so big that the human labor crowding feedback $\kappa$ (through wages?) reverses the effect on $M/H$.
\end{enumerate}

Here $\kappa$ is a tightness index for human labor market relative to skill-adjusted time requirement(?):
\begin{itemize}
    \item If $\kappa \approx 0$ very little of the available human labor capacity is being used: adding more $H$ hardly changes wages or the cutoff.
    \item If $\kappa$ is large, the human labor market is close to capacity: small increases in $H$ creates large upward pressure on human labor wage $w_H$, which strongly increases the threshold $u$.
\end{itemize}


\end{document}