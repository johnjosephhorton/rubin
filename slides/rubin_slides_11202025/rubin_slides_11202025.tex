%% LyX 2.4.2.1 created this file.  For more info, see https://www.lyx.org/.
%% Do not edit unless you really know what you are doing.
\documentclass[english,aspectratio=1610]{beamer}
\usepackage[T1]{fontenc}
\usepackage[latin9]{inputenc}
\setcounter{secnumdepth}{3}
\setcounter{tocdepth}{3}
\usepackage{dsfont}
\usepackage{amstext}
\usepackage{amsthm}
\PassOptionsToPackage{normalem}{ulem}
\usepackage{ulem}

\makeatletter
%%%%%%%%%%%%%%%%%%%%%%%%%%%%%% Textclass specific LaTeX commands.
% this default might be overridden by plain title style
\newcommand\makebeamertitle{\frame{\maketitle}}%
% (ERT) argument for the TOC
\AtBeginDocument{%
  \let\origtableofcontents=\tableofcontents
  \def\tableofcontents{\@ifnextchar[{\origtableofcontents}{\gobbletableofcontents}}
  \def\gobbletableofcontents#1{\origtableofcontents}
}
\theoremstyle{plain}
\newtheorem{thm}{\protect\theoremname}[section]
\theoremstyle{definition}
\newtheorem{defn}[thm]{\protect\definitionname}
\theoremstyle{plain}
\newtheorem{prop}[thm]{\protect\propositionname}
\ifx\proof\undefined
\newenvironment{proof}[1][\protect\proofname]{\par
	\normalfont\topsep6\p@\@plus6\p@\relax
	\trivlist
	\itemindent\parindent
	\item[\hskip\labelsep\scshape #1]\ignorespaces
}{%
	\endtrivlist\@endpefalse
}
\providecommand{\proofname}{Proof}
\fi

%%%%%%%%%%%%%%%%%%%%%%%%%%%%%% User specified LaTeX commands.
\theoremstyle{theorem}

\newtheorem{assumption}{Assumption}
\newtheorem{remark}{Remark}
\newtheorem{step}{Step}

\newtheorem{proposition}{Proposition}

\usepackage{tikz} 
\usepackage{wrapfig}
\usepackage{caption}
\setbeamertemplate{caption}[numbered]
\usepackage[position=top]{caption,subfig}
%\setbeamertemplate{navigation symbols}{}
\setbeamertemplate{footline}{\insertframenumber{}}

\AtBeginSection[]
{\begin{frame}{Outline}
\tableofcontents[currentsection] 
\end{frame} 
}


\usetheme{CambridgeUS}           % main theme
\usecolortheme{beaver}         % colour palette
\usefonttheme{professionalfonts} % use your document fonts
\setbeamertemplate{section in toc}[sections numbered] % tweak templates



%%%%%%%%%%%%%%%%%%%%
% For illustration of AI task automation flow
\usetikzlibrary{shapes.geometric, arrows, positioning, arrows.meta, fit, decorations.pathreplacing, calc}

% Define styles
\tikzstyle{task} = [rectangle, minimum width=1cm, minimum height=1cm, text centered, draw=black, fill=gray!10, rounded corners, text width=2.2cm, align=center]
\tikzstyle{decision} = [rectangle, minimum width=4cm, minimum height=2cm, text centered, draw=black, fill=orange!30, rounded corners, text width=3cm, align=center]
\tikzstyle{decision2} = [rectangle, minimum width=4cm, minimum height=2cm, text centered, draw=black, fill=green!30, rounded corners, text width=3cm, align=center]
\tikzstyle{process} = [rectangle, minimum width=2.5cm, minimum height=2cm, text centered, draw=black, fill=olive!30, rounded corners, text width=3cm, align=center]
\tikzstyle{process2} = [rectangle, minimum width=2.5cm, minimum height=2cm, text centered, draw=black, fill=blue!30, text width=3cm, align=center]
\tikzstyle{smallbox} = [rectangle, minimum width=1cm, minimum height=1cm, text centered, draw=black, fill=olive!30, text width=1.5cm, align=center]
\tikzstyle{arrow} = [thick,->,>=stealth]
%%%%%%%%%%%%%%%%%%%%



\newcommand{\T}[0]{\mathcal{T}}
\newcommand{\J}[0]{\mathcal{J}}
\newcommand{\machine}[1]{\langle #1 \rangle}
\newcommand{\human}[1]{( #1 )}
\newcommand{\cost}[1]{C\{ #1 \}}
\newcommand{\costdo}[1]{T_H\{ #1 \}}
\newcommand{\costmanage}[1]{T_M\{ #1 \}}
\newcommand{\timecost}[1]{t_{#1}}
\newcommand{\hccost}[1]{c_{#1}}
\newcommand{\labor}[1]{l_{#1}}
\newcommand{\handofftime}[1]{t^{s}_{#1}}
\newcommand{\humantime}[1]{t^{h}_{#1}}
\newcommand{\machinetime}[1]{t^{m}_{#1}}
\newcommand{\humanhc}[1]{c^{h}_{#1}}
\newcommand{\machinehc}[1]{c^{m}_{#1}}

\makeatother

\usepackage{babel}
\providecommand{\definitionname}{Definition}
\providecommand{\propositionname}{Proposition}
\providecommand{\theoremname}{Theorem}

\begin{document}
\title[Economic Impacts of GenAI on Structure of Work]{The Economic Impacts of Generative AI on the Structure of Work}
\author[Demirer, Horton, Immorlica, Lucier, Shahidi (2025)]{Mert Demirer (MIT)\\
John J. Horton (MIT)\\
\textbf{Nicole Immorlica (Yale \& Microsoft)}\\
Brendan Lucier (Microsoft)\\
Peyman Shahidi (MIT)\textbf{}\\
\textbf{}\\
\vspace{0.9cm}
{[}Your Conference Title Here{]}}
\date{{[}Date Here{]}}

\makebeamertitle
\section{Introduction}

\begin{frame}{Motivation}

\begin{itemize}
\item Hot question of the day: how Artificial Intelligence (AI) will affect
human workers?
\begin{itemize}
\item Will it substitute or complement them? Raise or reduce wages? Upskill
or deskill workers? \pause 
\end{itemize}
\item Many recent papers studied this, often drawing on established approaches.\begin{figure}[h!]
 \begin{center}
   \includegraphics[width=0.6\textwidth]{plots/p.png}
 \end{center}
\end{figure}\pause
\item Though the tools they use not originally designed to explain feature
of GenAI in mind.
\end{itemize}
\end{frame}
%
\begin{frame}[noframenumbering]{Motivation}

\begin{itemize}
\item Hot question of the day: how Artificial Intelligence (AI) will affect
human workers?
\begin{itemize}
\item Will it substitute or complement them? Raise or reduce wages? Upskill
or deskill workers?
\end{itemize}
\item Many recent papers studied this, often drawing on established approaches.
\begin{figure}[h!]
 \begin{center}
   \includegraphics[width=0.6\textwidth]{plots/pp.png}
 \end{center}
\end{figure}
\item Except for a few...
\end{itemize}
\end{frame}
%
\begin{frame}{Paper in a Nutshell}
\begin{itemize}
\item In this paper, we: 
\begin{itemize}
\item Develop a task-based production framework tailored to features of
GenAI,
\item Study effect of AI on firm's organizational structure, treating tasks
and jobs as endogenous objects,
\item Map firm-level production functions to a macro CES production function,
assuming firms deploy the same general-purpose AI technology differently,
and
\item Empirically validate several implications of the model. \pause
\end{itemize}
\item In our framework:
\begin{itemize}
\item Three modes of task execution recognized: manual, AI-augmented, AI-automated,
\item AI can chain multiple tasks together, \textit{potentially} overturning
standard comparative-advantage logic,
\item AI can either upskill or deskill workers,
\item Improvements in AI quality generate discontinuous productivity gains
(by triggering discrete reorganizations of work).
\end{itemize}
\item The paper relates to organization and labor economics literatures,
and economics of of AI.
\end{itemize}
\end{frame}
%
\section{Model}
\begin{frame}{A Task-based Model of Production}
\begin{itemize}
\item Production is a sequence of steps (which later are aggregated to tasks,
and then to jobs).
\item \uline{All} steps have to be completed to produce the good $\rightarrow$
production is Leontief in steps.
\item Consider, e.g., data scientists workflow:
\begin{itemize}
\item Step 1: Define the business decision and success metric.
\item Step 2: Discover and fetch relevant data for answering the question.
\item Step 3: Build an analysis pipeline (e.g., write code).
\item Step 4: Produce report and visuals based on analysis outputs.
\item Step 5: Present to and persuade supervisor to take action.
\end{itemize}
\end{itemize}
\begin{figure}[h!]
  \begin{center}
\resizebox{0.75\textwidth}{!}{
\begin{tikzpicture}[node distance=1.8cm]

    % Task Sequence
    \node (task1) [task] {Step 1};
    \node (task2) [task, right=1.75cm of task1] {Step 2};
    \node (task3) [task, right=1.75cm of task2] {Step 3};
    \node (task4) [task, right=1.75cm of task3] {Step 4};
    \node (task5) [task, right=1.75cm of task4] {Step 5};

    % Arrows
    \draw [arrow] (task1.east) -- 
    %node[midway, above] {$\humantime{} + \handofftime{}$} 
    (task2.west);
    \draw [arrow] (task2.east) -- 
    %node[midway, above] {$\humantime{} + \handofftime{}$} 
    (task3.west);
    \draw [arrow] (task3.east) -- 
    %node[midway, above] {$\humantime{} + \handofftime{}$} 
    (task4.west);
    \draw [arrow] (task4.east) -- 
    %node[midway, above] {$\humantime{} + \handofftime{}$} 
    (task5.west);
\end{tikzpicture}}
   \end{center}
\end{figure}
\end{frame}
%
\begin{frame}{Geometric Representation of Steps and Production}
\begin{itemize}
\item Steps have intrinsic cost parameters in 2(+1) dimensions:
\begin{enumerate}
\item Skill $c_{i}$ $\rightarrow$ human capital required to perform the
step
\item Time $t_{i}$ $\rightarrow$ time it takes to complete the step
\end{enumerate}
\item Production can be viewed as stacked rectangles of height $c_{i}$
and width $t_{i}$:
\end{itemize}
\begin{figure}[h!]
 \begin{center}
   \includegraphics[width=0.34\textwidth]{plots/tent_poll.png}
 \end{center}
\end{figure}
\end{frame}
%
\begin{frame}{Step Execution in Absence of AI}
\begin{itemize}
\item In absence of AI, all steps are executed manually.
\end{itemize}
\begin{figure}[h!]
  \begin{center}
\resizebox{0.88\textwidth}{!}{
\begin{tikzpicture}[node distance=1.8cm]

    % Task Sequence
    \node (task1) [task] {Step 1};
    \node (task2) [task, right=1.75cm of task1] {Step 2};
    \node (task3) [task, right=1.75cm of task2] {Step 3};
    \node (task4) [task, right=1.75cm of task3] {Step 4};
    \node (task5) [task, right=1.75cm of task4] {Step 5};

    % Arrows
    \draw [arrow] (task1.east) -- 
    %node[midway, above] {$\humantime{} + \handofftime{}$} 
    (task2.west);
    \draw [arrow] (task2.east) -- 
    %node[midway, above] {$\humantime{} + \handofftime{}$} 
    (task3.west);
    \draw [arrow] (task3.east) -- 
    %node[midway, above] {$\humantime{} + \handofftime{}$} 
    (task4.west);
    \draw [arrow] (task4.east) -- 
    %node[midway, above] {$\humantime{} + \handofftime{}$} 
    (task5.west);
\end{tikzpicture}}
   \end{center}
\end{figure}
\end{frame}
%
\begin{frame}{Definition - Manual Step}
\begin{defn}[Manual Step]
 A step $i$ is said to be performed manually if it is executed entirely
by a human worker without AI assistance. The associated skill and
time costs for a manual step are denoted by $(c_{i}^{H},t_{i}^{H})$.
\end{defn}
\end{frame}
%
\begin{frame}{AI Can Augment Humans in Performing Step}
\begin{itemize}
\item AI can assist humans in executing steps.\begin{figure}[h!]
  \begin{center}
\resizebox{0.88\textwidth}{!}{
\begin{tikzpicture}[node distance=1.8cm]

    % Task Sequence
    \node (task1) [task] {Step 1};
    \node (task2) [task, right=1.75cm of task1] {Step 2};
    \node (task3) [task, right=1.75cm of task2] {Step 3};
    \node (task4) [decision, right=1.75cm of task3] {Step 4: \\ \textbf{Can AI help with this step?}};
    \node (task5) [task, right=1.75cm of task4] {Step 5};

    % AI Processes
    \node (prompt) [process, below=2.5cm of task2, xshift=-1cm] {Ask AI to do Step 4 \\ (``prompt'')};
    \node (attempt) [process2, right=0.75cm of prompt] {AI produces an output};
    \node (evaluate) [process, right=0.75cm of attempt] {Is the work acceptable?};

    % Define a coordinate 1cm to the right of the evaluate box
    \coordinate (extendEval) at ([xshift=1cm]evaluate.east);

    % Success/Failure Nodes
    \node (yes) [smallbox, right=1cm of extendEval, yshift=1cm] {Yes};
    \node (no) [smallbox, right=1cm of extendEval, yshift=-1cm] {No};

    % Arrows
    \draw [arrow] (task1.east) -- 
    %node[midway, above] {$\humantime{} + \handofftime{}$} 
    (task2.west);
    \draw [arrow] (task2.east) -- 
    %node[midway, above] {$\humantime{} + \handofftime{}$} 
    (task3.west);
    \draw [arrow] (task3.east) -- 
    %node[midway, above] {$\humantime{} + \handofftime{}$} 
    (task4.west);
    \draw [arrow] (task4.south) to[out=-125,in=45] 
    %node[midway, above] {$\timecost{p}$} 
    (prompt.north);
    \draw [arrow] (prompt.east) -- (attempt.west);
    \draw [arrow] (attempt.east) -- (evaluate.west);
    
    % Draw extension line before branching
    \draw [arrow] (evaluate.east) -- 
    %node[midway, above] {$\timecost{e}$} 
    (extendEval);
    
    % Arrows from the extended coordinate
    \draw [arrow] (extendEval) -- node[midway, above] {$q_4 \ \ \ \ $} (yes.west);
    \draw [arrow] (extendEval) -- node[midway, below] {$1-q_4 \ \ \ \ \ $} (no.west);

    \draw [arrow] (yes.east) to[out=0,in=-90] 
    %node[midway, right] {$\ \ \ \ \ \handofftime{}$} 
    (task5.south);
    \draw [arrow] (no.south) to[out=-165,in=-30] (prompt.south);

\end{tikzpicture}}
   \end{center}
\end{figure}Step 4: Produce report and visuals based on analysis outputs.
\end{itemize}
\end{frame}
%
\begin{frame}{Definition - Augmented Step}
\begin{defn}[Augmented Step]
 A step $i$ is said to be augmented if it is executed by the AI,
after which its output is reviewed and approved by a human worker.
The associated skill and (expected) time costs of managing a single
augmented step are denoted by $(c_{i}^{M},\text{\ensuremath{\tfrac{t_{i}^{M}}{q_{i}}}})$,
where $q_{i}\in(0,1]$ is the AI's probability of successfully completing
the step.
\end{defn}
\begin{itemize}
\item Specifically, $q_{i}=\alpha^{d_{i}}$:
\begin{itemize}
\item $\alpha\in(0,1]$ is the general-purpose AI's quality level: $\uparrow\alpha\longrightarrow\uparrow q_{i}$,
\item $d_{i}>0$ is AI's difficulty in performing step $i$: $\uparrow d_{i}\longrightarrow\downarrow q_{i}$.
\end{itemize}
\end{itemize}
\end{frame}
%
\begin{frame}{AI Can Also Automate Steps}
\begin{itemize}
\item AI can also execute steps without direct human oversight in ``AI
Chains.''\begin{figure}[h!]
  \begin{center}
\resizebox{0.88\textwidth}{!}{
\begin{tikzpicture}[
    node distance=1.8cm,
    dashedbox/.style={draw, rectangle, rounded corners, dashed, inner sep=0.5cm, line width=1.2pt}
]

	% Task Sequence
	\node (task1) [task] {Step 1};
	\node (task2) [decision2, right=1.75cm of task1] {Step 2: \\ \textbf{Can AI do this ``under the hood''?}};
	\node (task3) [decision2, right=1.75cm of task2] {Step 3: \\ \textbf{Can AI do this ``under the hood''?}};
	\node (task4) [decision, right=1.75cm of task3] {Step 4: \\ Can AI help with this step?};
	\node (task5) [task, right=1.75cm of task4] {Step 5};

	% Dashed group around Steps 2-4
	\node[dashedbox, draw=blue, fit=(task2)(task3)(task4)] (AIchain) {};

    % AI Processes
    \node (prompt) [process, below=2.5cm of task2, xshift=-1cm] {Ask AI to do \textcolor{red}{Steps 2-4} \\ (``prompt'')};
    \node (attempt) [process2, right=0.75cm of prompt] {AI produces an output};
    \node (evaluate) [process, right=0.75cm of attempt] {Is the work acceptable?};

    % Define a coordinate 1cm to the right of the evaluate box
    \coordinate (extendEval) at ([xshift=1cm]evaluate.east);

    % Success/Failure Nodes
    \node (yes) [smallbox, right=1cm of extendEval, yshift=1cm] {Yes};
    \node (no) [smallbox, right=1cm of extendEval, yshift=-1cm] {No};

    % Arrows
    \draw [arrow] (task1.east) -- 
    %node[midway, above] {$\humantime{} + \handofftime{}$} 
    (task2.west);
    \draw [arrow] (task2.east) -- 
    %node[midway, above] {$\humantime{} + \handofftime{}$} 
    (task3.west);
    \draw [arrow] (task3.east) -- 
    %node[midway, above] {$\humantime{} + \handofftime{}$} 
    (task4.west);
    \draw [arrow] (task4.south) to[out=-125,in=45] 
    %node[midway, above] {$\timecost{p}$} 
    (prompt.north);
    \draw [arrow] (prompt.east) -- (attempt.west);
    \draw [arrow] (attempt.east) -- (evaluate.west);
    
    % Draw extension line before branching
    \draw [arrow] (evaluate.east) -- 
    %node[midway, above] {$\timecost{e}$} 
    (extendEval);
    
    % Arrows from the extended coordinate
    \draw [arrow] (extendEval) -- node[midway, above] {$\textcolor{red}{\tilde{q}} \ \ \ \ $} (yes.west);
    \draw [arrow] (extendEval) -- node[midway, below] {$\textcolor{red}{1-\tilde{q}}\  \ \ \ \ $} (no.west);

    \draw [arrow] (yes.east) to[out=0,in=-90] 
    %node[midway, right] {$\ \ \ \ \ \handofftime{}$} 
    (task5.south);
    \draw [arrow] (no.south) to[out=-165,in=-30] (prompt.south);

\end{tikzpicture}}
   \end{center}
\end{figure}

Step 2: Discover and fetch relevant data for answering the question.

Step 3: Build an analysis pipeline (e.g., write code).

Step 4: Produce report and visuals based on analysis outputs.
\end{itemize}
\end{frame}
%
\begin{frame}{Definition - Automated Step and AI Chain}
\begin{defn}[Automated Step]
 A step is said to be automated if it is executed entirely by AI
without direct human intervention, and its output is passed directly
to a subsequent augmented or automated step. The direct human costs
(in both skill and time dimension) associated with an automated step
are zero.
\end{defn}
\vspace{0.5cm}
\begin{defn}[AI Chain]
 An AI chain is a contiguous block of one or more sequential steps
executed by AI, in which all steps but the final one are automated
and the final step is augmented. An AI chain spanning steps $(s_{\ell},\dots,s_{i})$
has steps $(s_{\ell},\dots,s_{i-1})$ automated and step $s_{i}$
augmented. The skill and time costs of this AI chain are given by
$(c_{i}^{M},\tfrac{t_{i}^{M}}{\prod_{r=\ell}^{i}q_{r}})$, where $q_{r}$
denotes the AI success probability for step $s_{r}$.
\end{defn}
\end{frame}
%
\begin{frame}{Three Modes of Step Execution}
\begin{itemize}
\item The model recognizes three modes of step execution: 
\begin{enumerate}
\item Manually by human (Steps 1 and 5)
\item Augmented with AI but still requiring human oversight (Step 4)
\item Fully Automated by AI without \uline{direct} human oversight (Steps
2 and 3)
\end{enumerate}
\end{itemize}
\begin{figure}[ht!]
  \begin{center}
  \begin{tikzpicture}[scale=0.75, transform shape,
    node distance=0.5cm and 1.2cm,
    every node/.style={rectangle, rounded corners, draw, align=center, minimum width=2cm, minimum height=1cm},
    manual/.style={fill=gray!10},
    automated/.style={fill=green!30},
    augmented/.style={fill=orange!30},
    labelbox/.style={draw, rectangle, rounded corners, font=\bfseries, minimum width=2.25cm, minimum height=1cm},
    dashedbox/.style={draw, rectangle, rounded corners, dashed, inner sep=0.3cm, line width=1pt},
    >=latex
  ]

    % Top layer (Steps)
    \node[manual] (S1) {Step 1\\(Manual)};
    \node[automated, right=of S1] (S2) {Step 2\\(Automated)};
    \node[automated, right=of S2] (S3) {Step 3\\(Automated)};
    \node[augmented, right=of S3] (S4) {Step 4\\(Augmented)};
    \node[manual, right=of S4] (S5) {Step 5\\(Manual)};

    % Arrows between steps
    \draw[->, thick] (S1) -- (S2);
    \draw[->, thick] (S2) -- (S3);
    \draw[->, thick] (S3) -- (S4);
    \draw[->, thick] (S4) -- (S5);

    % Dashed boxes around steps
    \node[dashedbox, draw=olive, fit=(S1)] (DS1) {};
    \node[dashedbox, draw=blue, fit=(S2)(S3)(S4)] (DS2-4) {};
    \node[dashedbox, draw=olive, fit=(S5)] (DS5) {};

    % AI box top center aligned with Step 3
    \node[labelbox, fill=blue!20, above=1.25cm of S3] (AI) {AI};

    % Human box bottom center aligned with Step 3
    \node[labelbox, fill=olive!20, below=1.5cm of S3] (Human) {Human};

    % Lines from AI box to AI steps
    \draw[thin, blue!80] (AI.south) -- (S2.north);
    \draw[thin, blue!80] (AI.south) -- (S3.north);
    \draw[thin, blue] (AI.south) -- (S4.north);

    % Lines from Human box to Human steps
    \draw[thin, olive!80] (S1.south) -- (Human.north);
    \draw[thin, olive] (S4.south) -- (Human.north);
    \draw[thin, olive!80] (S5.south) -- (Human.north);
  \end{tikzpicture}
  \end{center}
\end{figure}
\end{frame}
%
\begin{frame}{Model}
\begin{itemize}
\item Firm's production consists of a sequence of steps. Steps are partitioned
into tasks ($\textit{\textit{automation strategy}}$), and tasks are
partitioned into jobs ($\textit{job design}$).
\end{itemize}
\begin{figure}[htbp]
    \begin{center}
    \includegraphics[width=\linewidth]{plots/hierarchical_production_illustration.png}
    \end{center}
\end{figure}
\end{frame}
%
\begin{frame}{Notation and Other Assumptions}
\begin{itemize}
\item Firm's production consists of a sequence of steps $\mathcal{S}=(s_{1},\ldots,s_{m})$.
\item Firm partitions contiguous subsequences of steps into a sequence of
tasks $\mathcal{T}=(T_{1},\ldots,T_{n})$ where $T_{b}=(s_{i},\ldots,s_{i+\ell})$
for some $i$ and $\ell$.
\item Firm also partitions contiguous subsequences of tasks into jobs $\mathcal{J}=(J_{1},\ldots,J_{p}).$
\item We call $\mathcal{T}$ the firm's $\textit{\textit{automation strategy}}$
and $\mathcal{J}$ the firm's $\textit{job design}$.
\end{itemize}
$\ $
\begin{itemize}
\item Labor is the only input (for now).
\item Firm assigns each job to a single worker.
\item Workers are ex-ante identical; but acquire human capital needed to
perform skills associated with tasks in their jobs after assignment.
\end{itemize}
\end{frame}
%
\begin{frame}{Construction of Tasks from Steps, and Jobs from Tasks}
\begin{itemize}
\item Tasks are the basic units of work of a human worker. 
\item A task is either a human task or an AI chain. Task's cost derived
from its constituent steps: 
\begin{itemize}
\item $(c_{i}^{H},t_{i}^{H})$if human,
\item $\left(c_{i}^{M},\frac{t_{i}^{M}}{\prod_{r=\ell}^{i}q_{r}}\right)$
if AI chain.
\end{itemize}
\end{itemize}
\vspace{0.5cm}
\begin{itemize}
\item A contiguous set of tasks are aggregated into a job. Cost parameters
of job $j$ determined by its constituent tasks:
\begin{itemize}
\item $\text{Wage}_{j}=\sum_{\text{task }b\in j}c_{b}$
\item $\text{\ensuremath{\text{Employment Duration}_{j}}}=\sum_{\text{task }b\in j}t_{b}.$
\end{itemize}
\item Workers paid wage rates for every unit of time being employed. Thus,
wage bill of job $j$ is:
\[
\text{WageBill}_{j}=\left(\sum_{\text{task }b\in j}c_{b}\right)\left(\sum_{\text{task }b\in j}t_{b}\right).
\]
\textcolor{white}{The term $t^{S}(j)$ is the hand-off time cost associated
with the last task of the job (and hence the last step of the job).}
\end{itemize}
\end{frame}
%
\begin{frame}{Visual Example of Job Construction}

\begin{figure}[htbp]
    \begin{center}
    \includegraphics[width=0.65\linewidth]{plots/combined_grid_no_handoff.png}
    \end{center}
\end{figure}
\end{frame}
%
\begin{frame}{Construction of Jobs from Tasks - Hand-off Cost Introduction}
\begin{itemize}
\item Tasks are the basic units of work of a human worker. 
\item A task is either a human task or an AI chain. Task's cost derived
from its constituent steps: 
\begin{itemize}
\item $(c_{i}^{H},t_{i}^{H})$if human,
\item $\left(c_{i}^{M},\frac{t_{i}^{M}}{\prod_{r=\ell}^{i}q_{r}}\right)$
if AI chain.
\end{itemize}
\end{itemize}
\vspace{0.5cm}
\begin{itemize}
\item A contiguous set of tasks are aggregated into a job. Cost parameters
of job $j$ determined by its constituent tasks:
\begin{itemize}
\item $\text{Wage}_{j}=\sum_{\text{task }b\in j}c_{b}$
\item $\text{\ensuremath{\text{Employment Duration}_{j}}}={\color{red}t^{S}(j}\mathclose{\color{red})}+\sum_{\text{task }b\in j}t_{b}.$
\end{itemize}
\item Workers paid wage rates for every unit of time being employed. Thus,
wage bill of job $j$ is:
\[
\text{WageBill}_{j}=\left(\sum_{\text{task }b\in j}c_{b}\right)\left({\color{red}t^{S}(j}\mathclose{\color{red})}+\sum_{\text{task }b\in j}t_{b}\right).
\]
The term \textcolor{red}{$t^{S}(j)$} is the hand-off time cost associated
with the last task of the job.
\end{itemize}
\end{frame}
%
\begin{frame}{Aggregation of Steps into Tasks into Jobs with Hand-offs}
\begin{itemize}
\item Hand-off only realized at the boundary of jobs (traced back from jobs
layer to steps layer).
\item Hand-off time cost is an intrinsic property of steps (the third cost
parameter).
\end{itemize}
\begin{figure}[htbp]
    \begin{center}
    \includegraphics[width=\linewidth]{plots/hierarchical_production_illustration.png}
    \end{center}
\end{figure}
\end{frame}
%
\begin{frame}{Role of Hand-off in Determining Job Boundaries}
\begin{itemize}
\item Hand-off captures inefficiencies arising from coordination frictions
among multiple workers.
\item Firm trades off:
\begin{enumerate}
\item Specializing workers and paying a coordination cost,
\item Assigning more tasks to workers and paying them higher per-unit-time
wages but saving up on coordination costs.
\[
\text{TotalCost}=\sum_{j=1}^{J}\text{WageBill}_{j}=\sum_{j=1}^{J}\left[\left(\sum_{\text{task }b\in j}c_{b}\right)\left(t^{S}(j)+\sum_{\text{task }b\in j}t_{b}\right)\right].
\]
\end{enumerate}
\end{itemize}
\begin{figure}[htbp]
    \begin{center}
    \includegraphics[width=0.65\linewidth]{plots/job_design.png}
    \end{center}
\end{figure}
\end{frame}
%
\begin{frame}{Visual Example of Job Construction with Hand-offs}

\begin{figure}[htbp]
    \begin{center}
    \includegraphics[width=0.65\linewidth]{plots/combined_grid_with_handoff.png}
    \end{center}
\end{figure}
\end{frame}
%
\begin{frame}{Firm's Organizational Structure}
\begin{itemize}
\item Two optimization horizons:
\begin{enumerate}
\item Long-run: job design $\mathcal{J}$ and automation strategy $\mathcal{T}$
can flexibly adjust:
\[
\min_{\mathcal{T}}\ \min_{\mathcal{J}}\ \text{TotalCost}(\mathcal{J};\mathcal{T})=\sum_{J_{j}\in\mathcal{J}}\text{WageBill}_{j}=\sum_{J_{j}\in\mathcal{J}}\Biggl[\Biggl(\sum_{T_{b}\in J_{j}}c_{b}\Biggr)\Biggl(t^{S}(J_{j})+\sum_{T_{b}\in J_{j}}t_{b}\Biggr)\Biggr].
\]
\item Short-run: job design $\mathcal{J}$ and worker wages fixed, automation
strategy $\mathcal{T}$ adjusts within each job:
\[
\min_{\mathcal{T}}\;\sum_{T_{b}\in\mathcal{T}}t_{b}.
\]
\end{enumerate}
\end{itemize}
\end{frame}
%
\section{Optimization}
\begin{frame}{Short-Term Optimization: Automation Design}
\begin{itemize}
\item In the short-term, job boundaries and wages are fixed. AI only allows
to optimize completion times within jobs. 
\item W/o loss to assume there is a single job with $m$ steps.
\item Goal is to find the optimal automation strategy $\mathcal{T}$ that
minimizes total completion time.
\end{itemize}
\begin{figure}[htbp]
    \begin{center}
    \includegraphics[width=0.75\linewidth]{plots/hierarchical_production_illustration_shortTerm.png}
    \end{center}
\end{figure}
\end{frame}
%
\begin{frame}{Short-Term Optimization: Automation Design - Continued}
\begin{prop}
Given $m$ steps organized into a single job, the time-optimal automation
strategy can be calculated in time $O(m^{2})$ via dynamic programming.
\end{prop}
\begin{proof}
At step $k\leq m$ show the minimum time needed to complete the job
with $C[k]$. We have:
\[
C[k]=\min\ \Biggl\{ C[k-1]+t_{k}^{H},\ \min_{\ell<k}\Biggl\{ C[\ell]+\frac{t_{k}^{M}}{\prod_{i=\ell+1}^{k}q_{i}}\Biggr\}\Biggr\}.
\]
\begin{itemize}
\item First term: \textrm{Step $k$ done manually, optimize over steps 1
to $k-1$ separately}; 
\item Second term: \textrm{Step $k$ augmented; optimize the optimal length
of chain ending in $k$.}
\end{itemize}
Given $(C[0],C[1],...,C[k-1])$ we can calculate $C[k]$ in time $O(k)$
by considering each potential choice of $\ell$. Doing so for each
$k=1,2,...,m$ yields the value of $C[m]$ in $O(m^{2})$ time.
\end{proof}
\end{frame}
%
\begin{frame}{Approximate Solution to Long-Term Optimization Problem}
\begin{itemize}
\item Turns out, the long-term optimization is a massive combinatorial problem.
\hyperlink{A1:Optimization}{\beamergotobutton{Backup Slides}}
\item We approximate a near-optimal solution in polynomial time by discretizing
the skill and time costs:
\end{itemize}
\begin{prop}
Fix any sequence of $m$ steps $\mathcal{S}=(s_{1},\dotsc,s_{m})$
for which all skill costs, time costs, and hand-off costs lie in $[1/B,B]$
for some $B>0$. Then for any $\epsilon>0$, an approximately cost-minimizing
pair of automation strategy $\mathcal{T}$ and job design $\mathcal{J}$
minimizing the long-term production cost to within a factor of $(1+\epsilon)$
can be computed in time $O(m^{2}\epsilon^{-2}\log^{2}(mB))$ by dynamic
programming.
\end{prop}
\end{frame}
%
\section{Discussion}
\begin{frame}{Task Interdependence - Overturning of Comparative Advantage Logic}
\begin{itemize}
\item For a single step, the decision whether to execute manually or augment
with AI depends on which cost is lower.
\item However, a human-efficient step may be preferable to automate as a
part of an AI chain.
\end{itemize}
\begin{exampleblock}{Example 0}
Consider a production process with 2 steps with identical skill and
time cost parameters: $(c^{H},t^{H})=(1,\tfrac{3}{2})$ and $(c^{M},t^{M})=(1,1)$,
and take AI success probabilities $q_{1}=\tfrac{3}{5}$ and $q_{2}=\tfrac{4}{5}$.

It's more cost-efficient to execute Step 1 manually:
\[
\underbrace{(1)(\tfrac{3}{2})}_{\text{Manual Execution}}=\tfrac{3}{2}<\tfrac{5}{3}=\underbrace{(1)(\tfrac{3}{5})^{-1}}_{\text{AI-Augmented Execution}}.
\]

However, the AI chain has a lower cost than executing the first task
manually and augmenting the second step.
\[
\underbrace{(1)(1)(\tfrac{3}{5}\times\tfrac{4}{5})^{-1}}_{\text{AI Chain}}=\tfrac{25}{12}\ \ <\ \ \tfrac{11}{4}=\underbrace{(1)(\tfrac{3}{2})+(1)(\tfrac{4}{5})^{-1}}_{\text{Step 1 Manual, Step 2 AI-Augmented}}
\]
\end{exampleblock}
\end{frame}
%
\begin{frame}{Fragmentation Index - Motivating Examples}
\begin{itemize}
\item Because of task interdependence, it becomes important where AI-able
steps are positioned in the workflow.
\end{itemize}
\begin{exampleblock}{Example 1}
Consider a job consisting of $m(=2k)$ steps. Each step has $t^{H}=5$
and $t^{M}=1$, but the steps vary in how difficult they are for an
AI to complete: odd-numbered steps can be successfully completed by
an AI with probability 1, whereas even-numbered steps will be successfully
completed with probability only 0.1:

{\footnotesize
\[
\textcolor{blue!70}{(Easy)_{1}}\longrightarrow\mathopen{\color{red}(}{\color{red}Hard}\mathclose{\color{red})_{2}}\longrightarrow\textcolor{blue!70}{(Easy)_{3}}\longrightarrow\cdots\cdots\longrightarrow\mathopen{\color{red}(}{\color{red}Hard}\mathclose{\color{red})_{m-2}}\longrightarrow\textcolor{blue!70}{(Easy)_{m-1}}\longrightarrow\mathopen{\color{red}(}{\color{red}Hard}\mathclose{\color{red})_{m}}
\]
}{\footnotesize\par}

It is suboptimal to attempt to use AI on any of the even-numbered
steps, \textit{even as part of a chain}. The optimal automation strategy
is to employ AI-augmentation on the odd-numbered steps (for a cost
of 1 each) and perform even-numbered steps manually (for a cost of
5 each), resulting in overall time cost of $3m$.
\end{exampleblock}
\end{frame}
%
\begin{frame}{Fragmentation Index - Motivating Examples}
\begin{itemize}
\item Because of task interdependence, it becomes important where AI-able
steps are positioned in the workflow.
\end{itemize}
\begin{exampleblock}{Example 2}
Suppose in the previous example easy and hard steps are not interleaved,
but rather the first $\frac{m}{2}$ steps can each be completed by
AI with probability 1 and the last $\frac{m}{2}$ steps can each be
completed with probability 0.1:{\footnotesize
\[
\textcolor{blue!70}{(Easy)_{1}}\longrightarrow\textcolor{blue!70}{(Easy)_{2}}\longrightarrow\cdots\cdots\longrightarrow\textcolor{blue!70}{(Easy)_{\tfrac{m}{2}}}\longrightarrow\mathopen{\color{red}(}{\color{red}Hard}\mathclose{\color{red})_{\tfrac{m}{2}+1}}\longrightarrow{\cdots\cdots}\longrightarrow\mathopen{\color{red}(}{\color{red}Hard}\mathclose{\color{red})_{m-1}}\longrightarrow\mathopen{\color{red}(}{\color{red}Hard}\mathclose{\color{red})_{m}}
\]
}The optimal strategy chains together the first $\frac{m}{2}$ steps
into a single automated AI chain, for a combined cost of 1. The remaining
$\frac{m}{2}$ steps are performed manually. Overall time cost will
be $1+\frac{5m}{2}$, which is strictly less than $3m$ for $m\geq4$
steps.
\end{exampleblock}
\end{frame}
%
\begin{frame}{Fragmentation Index}
\begin{itemize}
\item To extend the idea, we introduce a measure of dispersion of AI-exposed
tasks in production process \uline{in the short run}: \textit{fragmentation index}.
$\uparrow$ FI $\rightarrow$ $\downarrow$ sensitive production is
to AI improvements. \hyperlink{A2:Fragmentation}{\beamergotobutton{Formal Definition}}
\item We show that the fragmentation index of a job approximates the time
cost of an optimal automation strategy for the job. 
\end{itemize}
\begin{prop}
Fix a single job with a sequence of $m$ steps $\mathcal{S}=\{s_{1},\dotsc,s_{m}\}$,
each with $t_{i}^{M}=1$. Let $FI$ denote its fragmentation index
and let $OPT$ denote the minimum time cost over all automation strategies.
Then $\tfrac{1}{8}OPT\leq FI\leq\tfrac{5}{4}OPT$. If we further assume
$t_{i}^{H}\geq1$ for all $i$, then $\tfrac{1}{4}OPT\leq FI\leq\tfrac{5}{4}OPT$.
\end{prop}
\begin{itemize}
\item Key takeaways: 
\begin{itemize}
\item Jobs with high fragmentation index will tend to have higher time costs
even under optimal use of AI chaining. (i.e., structure of production
limits returns to AI improvement).
\item Key driver of efficiency gains via AI automation not just the \# of
steps that are automated, but the \uline{number of adjacent steps}
in the production process that have high exposure to AI. \label{Fragmentation_Main}
\end{itemize}
\end{itemize}
\end{frame}
%
\begin{frame}{Impact of AI Deployment on Worker Specialization}
\begin{itemize}
\item In short-run, gains from AI derive primarily from cost-savings. \uline{But},
in the long run, automation may alter the skill requirements of workers
and degree of specialization.
\end{itemize}
\begin{exampleblock}{Example 3}
 Consider a job consisting of a single step. This step has manual
time and skill requirements $(c^{H},t^{H})=(5,5)$. Suppose that,
under AI augmentation, this step has $(c^{M},t^{M})=(1,1)$ and probability
$q=\tfrac{1}{8}$ of successful completion. Despite the task taking
longer to complete with AI (due to the low likelihood of success on
any individual attempt), the reduced skill requirement of managing
the AI process means that it is firm-optimal to employ AI augmentation.
\end{exampleblock}
\begin{itemize}
\item We allow skill and time requirements of work to vary endogenously,
and unify two views:
\begin{itemize}
\item AI automates mundane (i.e., low-skill, high-time) tasks and allows
high-skill workers to do more high-skill tasks.
\item AI deskills high-skilled labor and replaces them with low-skill AI
management.
\end{itemize}
\end{itemize}
\end{frame}
%
\begin{frame}{Ambiguous Effect on Worker Specialization}
\begin{itemize}
\item Direction of the effect? Our framework remains agnostic, and can support
both!
\item One hypothesis: AI automation has a normalizing effect on tasks, reverting
both skill and time requirements toward the mean (i.e., ``square-like).
\item If hand-off time requirements remain unchanged, then tent-pole inefficiencies
due to grouping of tasks together in the same job reduces $\rightarrow$
less worker specialization.
\end{itemize}
\begin{figure}[h!]
 \begin{center}
   \includegraphics[width=0.25\textwidth]{plots/tent_poll.png}
 \end{center}
\end{figure}
\end{frame}
%
\begin{frame}{Numerical Example Leading to Less Specialization}
\begin{exampleblock}{Example 4}
 Consider a two-step production process, with costs $(c_{1}^{H},t_{1}^{H},t_{1}^{S})=(2,4,5)$
and $(c_{2}^{H},t_{2}^{H})=(4,2)$. That is, the first step is low-skill
but time-intensive and the second step is high-skill but can be completed
quickly. Combining both steps into a single job (at a labor cost of
$(2+4)(4+2)=36$) is strictly worse than separating into two specialized
jobs with a handoff (at a total cost of $(2)(4+5)+(4)(2)=26$). 

However, if each step of production could be augmented separately
via AI to yield effective skill and time costs of $(c^{M},t^{M})=(2,2)$
for each, then the optimal design combines both augmented steps into
a single job at a total cost of $(2+2)(2+2)=16$, which is better
than separating into two distinct jobs (incurring cost $(2)(2+5)+(2)(2)=18$).
\end{exampleblock}
\end{frame}
%
\begin{frame}{Numerical Example Leading to More Specialization}
\begin{itemize}
\item If AI-augmented steps have higher skill requirements than the corresponding
manual steps.
\end{itemize}
\begin{exampleblock}{Example 5}
 Consider a different two-step production process, with manual skill
and time costs $(c_{1}^{H},t_{1}^{H})=(c_{2}^{H},t_{2}^{H})$ and
hand-off cost $t_{1}^{S}=5$.

The optimal job design separates these into two separate jobs at a
total cost of $(3)(3+5)+(3)(3)=33$, with each job requiring a worker
of skill $3$. 

Suppose AI augmentation can allow each step to be completed with $(c^{M},t^{M})=(2,1/4)$,
and an AI success probability of $1/8$, for a total expected execution
time of $(1/4)\times(1/8)^{-1}=2$. In this case, the optimal design
employs AI augmentation in each step and combines the two steps into
a single job, resulting in a total cost of $(2+2)(2+2)=16$ and requiring
a worker of skill $4$. 

This is preferable to combining the two steps into a single AI chain,
which would result in a total cost of $(2)((1/4)\times(1/8)^{-1}\times(1/8)^{-1})=32$.
\end{exampleblock}
\end{frame}
%
\begin{frame}{Non-monotone Impacts of AI Improvements}
\begin{exampleblock}{Example 6}
 Consider a production process with two steps. The first step is
short but high-skill and difficult for AI tools to perform correctly:
$(c_{1}^{H},t_{1}^{H},t_{1}^{S})=(5,1,1)$. Task 1's AI difficulty
score, $d_{1}$, is $6$. Therefore $q_{1}=\alpha^{6}$. The AI management
skill and time requirements are the same as the manual execution requirements:
$(c_{1}^{M},t_{1}^{M})=(5,1)$. The second step is low-skill and timely
to execute manually, but easy for AI: $(c_{2}^{H},t_{1}^{H})=(2,4)$
and $(c_{2}^{M},t_{2}^{M})=(2,1)$ with AI difficulty score $1$. 
\end{exampleblock}
\begin{figure}[h!]
 \begin{center}
   \includegraphics[width=0.8\textwidth]{plots/nonmonotone_returns.png}
 \end{center}
\end{figure}
\end{frame}
%
\section{Production Function Aggregation}
\begin{frame}{Micro to Macro Production Function Aggregation - High-level Summary}
\begin{itemize}
\item To say anything about broader impacts of AI need to have a macro production
function. \label{Aggregation_Main}
\item We aggregate individual firms' production functions into a macro CES
production function in two steps:
\begin{enumerate}
\item From
\[
x=\min\left\{ \frac{l_{1}}{\tau_{1}^{M}\,\alpha^{-d_{1}}},\,\cdots,\,\frac{l_{k}}{\tau_{k}^{M}\,\alpha^{-d_{k}}},\,\frac{l_{k+1}}{\tau_{k+1}^{H}},\,\cdots,\,\frac{l_{n}}{\tau_{n}^{H}},\,\frac{l_{n+1}}{\tau^{S}(J_{1})},\,\cdots,\,\frac{l_{n+p-1}}{\tau^{S}(J_{p-1})}\right\} ,
\]
to
\[
x=\min\,\left\{ \frac{\bar{\alpha}\,l_{M}}{\tau_{M}},\,\frac{l_{H}}{\tau_{H}}\right\} ,
\]
where $\tau_{M}$ and $\tau_{H}$ are firm-level aggregate management
and human skill-adjusted labor.
\item With ``proper'' heterogeneity across firms in AI deployment quality,
the individual production functions aggregate into:
\[
X=\left(\theta_{M}\,M^{\rho}+\theta_{H}\,H^{\rho}+(1-\theta_{M}-\theta_{H})\,K^{\rho}\right)^{\frac{1}{\rho}},
\]
where $M$ and $H$ are economy-wide aggregate management and human
labor. The capital $K$ can be normalized to $1$ without loss. \hyperlink{A3:Aggregation}{\beamergotobutton{Aggregation Details}}
\end{enumerate}
\end{itemize}
\end{frame}
%
\section{Empirical Evaluation}
\begin{frame}{Model's Predictions that We Test}
\begin{itemize}
\item We empirically validate three predictions of the model:
\begin{enumerate}
\item AI-executed steps cluster in contiguous blocks, forming AI chains.
\item Step-level AI exposure is not the only determinant of AI execution;
where AI-able steps sit in the workflow also matters for execution
outcomes. (Recall the fragmentation argument.)
\item The same step may be executed differently across occupations depending
on AI execution status of its neighbors.
\end{enumerate}
\end{itemize}
\end{frame}
%
\begin{frame}{Data Sources}
\begin{itemize}
\item We construct a task-level dataset by combining: \label{Data_Sources}
\begin{enumerate}
\item May 2023 O{*}NET tasks and occupations data
\begin{itemize}
\item We treat O{*}NET tasks as steps in our theoretical framework. 
\item With their mode of execution determined, steps are then mapped to
tasks.
\end{itemize}
\item Human-generated labels for AI exposure of tasks (Eloundou et al.,
2024)
\item AI execution outcomes of tasks from Anthropic's Economic Index dataset
(Handa et al., 2025)
\item GPT-produced sequence of tasks within occupation
\end{enumerate}
\end{itemize}
\vspace{0.25cm}
\begin{itemize}
\item (1)-(3) external sources. We create (4) by prompting GPT-5-mini to
order tasks. \hyperlink{A4:GPT_TaskOrderingPrompt}{\beamergotobutton{GPT Prompt}}
\end{itemize}
\end{frame}
%
\begin{frame}{Augmented and Automated Tasks (Anthropic)}
\begin{itemize}
\item Following approach of Anthropic who grouped Claude conversations into
5(+1) categories:
\end{itemize}
\begin{figure}[htbp]
    \begin{center}
    \includegraphics[width=0.9\linewidth]{plots/anthropic_executionMode_examples.png}
    \end{center}
\end{figure}
\end{frame}
%
\begin{frame}{Augmented and Automated Tasks - Merged w/ O{*}NET}
\begin{itemize}
\item After merging Anthropic labels with the O{*}NET set of tasks:
\end{itemize}
\begin{figure}[htbp]
    \begin{center}
    \includegraphics[width=0.75\linewidth]{plots/all_labels_histogram.png}
    \end{center}
\end{figure}
\end{frame}
%
\begin{frame}{Augmented and Automated Tasks - Example Task Sequence 1}
\begin{itemize}
\item Example occupation with many AI tasks:
\end{itemize}
\begin{figure}[htbp]
    \begin{center}
    \includegraphics[width=0.7\linewidth]{plots/task_sequence_15-1251.00_Computer_Programmers.png}
    \end{center}
\end{figure}
\end{frame}
%
\begin{frame}{Augmented and Automated Tasks - Example Task Sequence 2}
\begin{itemize}
\item Example occupation with some (out of order!) AI tasks:
\end{itemize}
\begin{figure}[htbp]
    \begin{center}
    \includegraphics[width=0.7\linewidth]{plots/task_sequence_11-3131.00_Training_and_Development_Managers.png}
    \end{center}
\end{figure}
\end{frame}
%
\begin{frame}{Augmented and Automated Tasks - Example Task Sequence 3}
\begin{itemize}
\item Example occupation with no AI tasks:
\end{itemize}
\begin{figure}[htbp]
    \begin{center}
    \includegraphics[width=0.7\linewidth]{plots/task_sequence_49-2096.00_Electronic_Equipment_Installers_and_Repairers__Motor_Vehicles.png}
    \end{center}
\end{figure}
\end{frame}
%
\begin{frame}{Pred. \#1: AI-executed Steps Tend to Form AI Chains}
\begin{itemize}
\item Prediction \#1: AI-executed steps tend to appear next together, forming
AI chains. \pause
\item We treat all AI steps the same (augmented or automated) and calculate:
\begin{itemize}
\item Average length of AI chains: 1.45
\item Average number of AI chains per occupation: 2.1 \pause
\end{itemize}
\end{itemize}
\vspace{0.25cm}
\begin{itemize}
\item What's the benchmark? Are these small or big?
\item Calculate same statistics in two different placebo datasets constructed
via:
\begin{enumerate}
\item Randomizing task positions within occupations,
\item Randomizing execution label assignments of tasks across entire dataset.
\end{enumerate}
\end{itemize}
\end{frame}
%
\begin{frame}{Pred. \#1: AI-executed Steps Tend to Form AI Chains - Robustness Test
1}
\begin{itemize}
\item Histogram of statistics from 1,000 different task-position reshuffles
vs. actual values\begin{figure}[htbp]
    \begin{center}
    \includegraphics[width=\linewidth]{plots/aiChain_length_count_taskPositionReshuffle.png}
    \end{center}
\end{figure}
\item Takeaways: 
\begin{itemize}
\item Occurrence of AI tasks in clusters in original data is meaningful,
\item GPT-generated occupation task sequence non-random.
\end{itemize}
\end{itemize}
\end{frame}
%
\begin{frame}{Pred. \#1: AI-executed Steps Tend to Form AI Chains - Robustness Test
2}
\begin{itemize}
\item Histogram of statistics from 1,000 different task execution label
reshuffles vs. actual values\begin{figure}[htbp]
    \begin{center}
    \includegraphics[width=\linewidth]{plots/aiChain_length_count_taskLabelReshuffle.png}
    \end{center}
\end{figure}
\item Takeaways: 
\begin{itemize}
\item Occurrence of AI tasks in clusters in original data is meaningful,
\item Non-random distribution of Anthropic-reported AI-executed tasks.
\end{itemize}
\end{itemize}
\end{frame}
%
\begin{frame}{Pred. \#2: AI-able Task Fragmentation Matters}
\begin{itemize}
\item Prediction \#2: Step-level AI exposure is not the only determinant
of AI execution; where AI-able steps sit in the workflow also matters
for execution outcomes. \pause
\item Regression:
\[
\text{ai\_fraction}_{o}=\beta_{0}+\beta_{1}\,\text{ai\_exposure}_{o}+\beta_{2}\,\text{fragmentation\_index}_{o}
\]

\begin{itemize}
\item $\text{ai\_fraction}_{o}$: fraction of AI-executed tasks in occupation
$o$
\item $\text{ai\_exposure}_{o}$: fraction of AI-exposed tasks (E1) in occupation
$o$
\item $\text{fragmentation\_index}_{o}$: measure of how fragmented AI tasks
are in occupation $o$
\end{itemize}
\item Model predicts $\beta_{1}>0$ and $\beta_{2}<0$.
\end{itemize}
\end{frame}
%
\begin{frame}{Pred. \#2: AI-able Task Fragmentation Matters - Measuring FI}
\begin{itemize}
\item Regression:
\[
\text{ai\_fraction}_{o}=\beta_{0}+\beta_{1}\,\text{ai\_exposure}_{o}+\beta_{2}\,\text{fragmentation\_index}_{o}
\]
\item Fragmentation index measured in four ways for robustness:
\begin{enumerate}
\item Based on realized AI execution outcomes, AI chain defined per model's
definition (strict)
\item Based on realized AI execution outcomes, AI chain defined regardless
of type of AI tasks -- augmented and automated tasks treated similarly
\item Based on AI exposure measures, only consider E1 tasks as AI-exposed
(strict)
\item Based on AI exposure measures, consider E1 and E2 tasks as AI-exposed
\end{enumerate}
\end{itemize}
\end{frame}
%
\begin{frame}{Pred. \#2: AI-able Task Fragmentation Matters - Results}

\[
\text{ai\_fraction}_{o}=\beta_{0}+\beta_{1}\,\text{ai\_exposure}_{o}+\beta_{2}\,\text{fragmentation\_index}_{o}
\]

\begin{figure}[htbp]
    \begin{center}
    \includegraphics[width=0.5\linewidth]{plots/fragmentation_regression.png}
    \end{center}
\end{figure}
\end{frame}
%
\begin{frame}[noframenumbering]{Pred. \#2: AI-able Task Fragmentation Matters - Results}

\[
\text{ai\_fraction}_{o}=\beta_{0}+\beta_{1}\,\text{ai\_exposure}_{o}+\beta_{2}\,\text{fragmentation\_index}_{o}
\]

\begin{figure}[htbp]
    \begin{center}
    \includegraphics[width=0.5\linewidth]{plots/fragmentation_regression_explanation1.png}
    \end{center}
\end{figure}
\end{frame}
%
\begin{frame}[noframenumbering]{Pred. \#2: AI-able Task Fragmentation Matters - Results}

\[
\text{ai\_fraction}_{o}=\beta_{0}+\beta_{1}\,\text{ai\_exposure}_{o}+\beta_{2}\,\text{fragmentation\_index}_{o}
\]

\begin{figure}[htbp]
    \begin{center}
    \includegraphics[width=0.5\linewidth]{plots/fragmentation_regression_explanation2.png}
    \end{center}
\end{figure}
\end{frame}
%
\begin{frame}[noframenumbering]{Pred. \#2: AI-able Task Fragmentation Matters - Results}

\[
\text{ai\_fraction}_{o}=\beta_{0}+\beta_{1}\,\text{ai\_exposure}_{o}+\beta_{2}\,\text{fragmentation\_index}_{o}
\]

\begin{figure}[htbp]
    \begin{center}
    \includegraphics[width=0.5\linewidth]{plots/fragmentation_regression_explanation3.png}
    \end{center}
\end{figure}
\end{frame}
%
\begin{frame}[noframenumbering]{Pred. \#2: AI-able Task Fragmentation Matters - Results}

\[
\text{ai\_fraction}_{o}=\beta_{0}+\beta_{1}\,\text{ai\_exposure}_{o}+\beta_{2}\,\text{fragmentation\_index}_{o}
\]

\begin{figure}[htbp]
    \begin{center}
    \includegraphics[width=0.5\linewidth]{plots/fragmentation_regression_explanation4.png}
    \end{center}
\end{figure}
\end{frame}
%
\begin{frame}{Pred. \#3: Execution Mode of Neighbors Matter in Step's Execution
Outcome}
\begin{itemize}
\item Prediction \#3: Same step may be executed differently across occupations
depending on AI execution status of its neighbors. \pause
\item Step X more likely to be automated in Occupation 1:\begin{figure}[htbp]
    \begin{center}
    \includegraphics[width=0.8\linewidth]{plots/sameTask_diffOcc_graphic.png}
    \end{center}
\end{figure}
\item Local gains from executing step X as part of an AI chain in Occupation
1 is higher.
\end{itemize}
\end{frame}
%
\begin{frame}{Pred. \#3: Execution Mode of Neighbors Matter in Step's Execution
Outcome - Practical Considerations}
\begin{itemize}
\item Issue: O{*}NET tasks are occupation-specific: no task appears in >1
occupation. \label{DWA}
\item Solution: O{*}NET has a hierarchical categorization of tasks.
\begin{itemize}
\item Multiple tasks $\rightarrow$ Single detailed work activity (DWA)
\end{itemize}
\item Tasks within a DWA are more or less similar. 
\begin{itemize}
\item To ensure similarity we further ask GPT-5-mini to pick the tasks within
DWAs that are most similar in terms of execution method and required
skills. \hyperlink{A4:GPT_DWAsimilarTasksPrompt}{\beamergotobutton{GPT Prompt}}
\end{itemize}
\end{itemize}
\end{frame}
%
\begin{frame}{Pred. \#3: Execution Mode of Neighbors Matter in Step's Execution
Outcome - Example DWAs}
\begin{itemize}
\item Example DWAs, and tasks dropped in the ``finding similar tasks''
procedure (red font):\begin{figure}[htbp]
    \begin{center}
    \includegraphics[width=\linewidth]{plots/sameDWA.png}
    \end{center}
\end{figure}
\end{itemize}
\end{frame}
%
\begin{frame}{Pred. \#3: Execution Mode of Neighbors Matter in Step's Execution
Outcome - Regression}
\begin{itemize}
\item Prediction \#3: Same step may be executed differently across occupations
depending on AI execution status of its neighbors.
\item Regression: (ran on the subset of ``similar'' tasks)
\begin{align*}
\Pr\!\left(\text{is\_automated}_{t}=1|X_{t}\right) & =\Lambda\!\Big(\beta_{0}+\beta_{1}\,\mathds{1}_{\{(t-2)\ \text{Task is AI}\}}+\beta_{2}\,\mathds{1}_{\{(t-1)\ \text{Task is AI}\}}\\
 & \ \ \ \ \ \ \ \ +\beta_{3}\,\mathds{1}_{\{(t+1)\ \text{Task is AI}\}}+\beta_{4}\,\mathds{1}_{\{(t+2)\ \text{Task is AI}\}}\Big)
\end{align*}

\begin{itemize}
\item $\Lambda$ is logistic CDF.
\item $\text{is\_automated}_{t}=1$ if task $t$ is automated.
\item $\mathds{1}_{\{(t-2)\ \text{Task is AI}\}}=1$ if task two positions
before task $t$ is AI.
\item $\mathds{1}_{\{(t-1)\ \text{Task is AI}\}}=1$ if task before task
$t$ is AI.
\item $\mathds{1}_{\{(t+1)\ \text{Task is AI}\}}=1$ if task after task
$t$ is AI.
\item $\mathds{1}_{\{(t+2)\ \text{Task is AI}\}}=1$ if task two positions
after task $t$ is AI.
\end{itemize}
\end{itemize}
\end{frame}
%
\begin{frame}{Pred. \#3: Execution Mode of Neighbors Matter in Step's Execution
Outcome - Results}
\begin{itemize}
\item Average marginal effect of immediate neighbors being AI on automatability
of middle task positive and robust.
\item Average marginal effect of distant neighbors being AI on automatability
of middle task weaker and disappear after occupational family FEs.
\begin{figure}[htbp]
    \begin{center}
    \includegraphics[width=0.65\linewidth]{plots/sameDWA_regression.png}
    \end{center}
\end{figure}
\end{itemize}
\end{frame}
%
\begin{frame}[noframenumbering]{Pred. \#3: Execution Mode of Neighbors Matter in Step's Execution
Outcome - Results}
\begin{itemize}
\item Average marginal effect of \uline{immediate neighbors} being AI on
automatability of middle task \uline{positive and robust}.
\item Average marginal effect of distant neighbors being AI on automatability
of middle task weaker and disappear after occupational family FEs.
\begin{figure}[htbp]
    \begin{center}
    \includegraphics[width=0.65\linewidth]{plots/sameDWA_regression_explanation1.png}
    \end{center}
\end{figure}
\end{itemize}
\end{frame}
%
\begin{frame}[noframenumbering]{Pred. \#3: Execution Mode of Neighbors Matter in Step's Execution
Outcome - Results}
\begin{itemize}
\item Average marginal effect of immediate neighbors being AI on automatability
of middle task positive and robust.
\item Average marginal effect of \uline{distant neighbors} being AI on automatability
of middle task \uline{weaker and disappear} after occupational family
FEs. \begin{figure}[htbp]
    \begin{center}
    \includegraphics[width=0.65\linewidth]{plots/sameDWA_regression_explanation2.png}
    \end{center}
\end{figure}
\end{itemize}
\end{frame}
%
\begin{frame}{Pred. \#3: Execution Mode of Neighbors Matter in Step's Execution
Outcome - Robustness Test}
\begin{itemize}
\item How should we interpret these? Could these have happened by chance?
\pause No!
\item Test: randomize position of tasks within occupations and repeat the
same analysis.
\item Compare position-randomized outcomes against actual outcomes.
\end{itemize}
\end{frame}
%
\begin{frame}{Pred. \#3: Execution Mode of Neighbors Matter in Step's Execution
Outcome - Robustness Test}
\begin{itemize}
\item Observation \#1: in random-position placebos all neighbors contribute
equally to automatability of middle task (i.e., all four distributions
similar mean and shape) \begin{figure}[htbp]
    \begin{center}
    \includegraphics[width=\linewidth]{plots/AME_filtered_is_automated_no_fe.png}
    \end{center}
\end{figure}
\end{itemize}
\end{frame}
%
\begin{frame}{Pred. \#3: Execution Mode of Neighbors Matter in Step's Execution
Outcome - Robustness Test}
\begin{itemize}
\item Observation \#2: effect of immediate neighbors (two middle columns)
stronger than in randomized-position placebos \begin{figure}[htbp]
    \begin{center}
    \includegraphics[width=\linewidth]{plots/AME_filtered_is_automated_no_fe.png}
    \end{center}
\end{figure}
\end{itemize}
\end{frame}
%
\begin{frame}{Pred. \#3: Execution Mode of Neighbors Matter in Step's Execution
Outcome - Robustness Test}
\begin{itemize}
\item Observation \#3: effect of farther neighbors (two side columns) weaker
than in randomized-position placebos \begin{figure}[htbp]
    \begin{center}
    \includegraphics[width=\linewidth]{plots/AME_filtered_is_automated_no_fe.png}
    \end{center}
\end{figure}
\end{itemize}
\end{frame}
%
\begin{frame}{Pred. \#3: Execution Mode of Neighbors Matter in Step's Execution
Outcome - Robustness Test}
\begin{itemize}
\item Observation \#4: effect of farther neighbors disappears (and even
becomes negative) after controlling for occupational families but
positive effect of immediate neighbors persists\begin{figure}[htbp]
    \begin{center}
    \includegraphics[width=0.8\linewidth]{plots/AME_filtered_is_automated_fe_MajorGroup.png}
    \end{center}
\end{figure}\begin{figure}[htbp]
    \begin{center}
    \includegraphics[width=0.8\linewidth]{plots/AME_filtered_is_automated_fe_MinorGroup.png}
    \end{center}
\end{figure}
\end{itemize}
\end{frame}
%
\begin{frame}[noframenumbering]{Pred. \#3: Execution Mode of Neighbors Matter in Step's Execution
Outcome - Robustness Test}
\begin{itemize}
\item Together: it is more likely that the same task is automated if its
immediate neighbors are AI-executed; farther neighbors do not contribute
to automation of task (at least not yet -- perhaps with more capable
AIs we see longer chains form and farther neighbors also contribute
to some extent)
\end{itemize}
\end{frame}
%
\section{Summary}
\begin{frame}{Summary}
\begin{itemize}
\item Developed a task-based framework to study implications of AI automation:
\begin{itemize}
\item Firm trades off coordination frictions against paying higher wages.
\item AI automation might overturn the simple comparative-advantage logic
through the chaining logic.
\item Gave efficient algorithms for the firm's cost minimization problem,
determining optimal organizational structure.
\end{itemize}
\item Showed that discrete reorganization of work leads to non-monotone
returns from improving AI quality.
\begin{itemize}
\item Explains why firms heavily invest in AI without immediate returns.
\end{itemize}
\item Showed how to map micro production functions to a CES macro production
function if firms heterogenous in how they deploy general-purpose
AI technology.
\item Provided empirical support for implications of the model.
\end{itemize}
\end{frame}
%
\begin{frame}[plain, noframenumbering]{}

\begin{center}

Thanks for your attention! \\~\\

{[}your email here{]}

\end{center}
\end{frame}
%
\section{Appendix}

\label{A1:Optimization}
\begin{frame}[noframenumbering]{Warm-up to Long-Term Optimization: Job Design Without AI}
\begin{prop}
Given a fixed sequence of tasks $\mathcal{T}=(T_{1},\dotsc,T_{n})$
with skill and time costs $c=(c_{1},\dotsc,c_{n})$ and $t=(t_{1},\dotsc,t_{n})$,
the cost-minimizing job design $\mathcal{J}$ can be computed in time
$O(n^{2})$ by dynamic programming.
\end{prop}
\begin{proof}
Let $C[k]$ be the minimum cost of a job design for tasks 1 through
$k$, \textit{including hand-off costs for task $k$}. We have:
\[
C[k]=\min_{s<k}\Biggl\{ C[s]+\Biggl[\Biggl(\sum_{i=s+1}^{k}c_{i}\Biggr)\Biggl(t_{k}^{S}+\sum_{i=s+1}^{k}t_{i}\Biggr)\Biggr]\Biggr\}.
\]
Optimize over the choice of final job $\{s+1,...,k\}$, accounting
recursively for the optimal job design for the remaining jobs. Each
$C[k]$ can be calculated in $O(k)$ time by considering each choice
of $s=1,...,k$. Doing so for each $k=1,2,...,n$ will take $O(n^{2})$
time.
\end{proof}
\end{frame}
%
\begin{frame}[noframenumbering]{Long-Term Optimization: Automation and Job Design}
\begin{itemize}
\item Let $V(i)$ denote the minimum cost required to complete steps 1 through
$i$, including any hand-off costs to a subsequent worker.
\item Let $W(i,c,t)$ be the minimum cost of completing steps 1 through
$i$, as well as some additional (but unspecified) set of tasks with
cost profile $(c,t)$ that have already been assigned as a single
job to a single worker referred to as the ``active worker''.
\item By definition:
\[
V(i)=W(i,0,t_{i}^{S})
\]
\end{itemize}
\end{frame}
%
\begin{frame}[noframenumbering]{Long-Term Optimization: Automation and Job Design - Continued}
\begin{prop}
For each $i\geq1$, the minimum cost function $W(i,\;c,\;t)$ satisfies
the following recursive relation:

\begin{align*}
W(i,\; \hccost{},\; \timecost{})
= \min\Biggl\{ 
& \hccost{}\,\timecost{} \; + \; V(i), \; \; \; \; W(i-1,\; \hccost{} + c_i ^H,\; \timecost{} + t_i ^H), \\
& \ \min_{r < i} \ W\left(r,\; \hccost{} + c_i ^M,\; \timecost{} + \frac{t_i^M}{\prod_{s=r+1}^{i} q_s}\right)
\Biggr\}.
\end{align*}

The cost of the optimal joint automation and job design for a given
sequence of $m$ steps is then $V(m)=W(m,\;0,\;0)$.
\end{prop}
\begin{itemize}
\item Term 1: end ``previous'' worker's job at $i$; start job of active
worker from $i+1$. Cost of steps 1 to $i$ is $V(i)$.
\item Term 2: step $i$ executed manually; its manual costs are added to
active worker's job costs.
\item Term 3: augment step $i$ and choose the length of AI chain task terminating
in $i$ optimally.
\end{itemize}
\hyperlink{Optimization_Main}{\beamergotobutton{Return}}
\end{frame}
%
\label{A2:Fragmentation}
\begin{frame}{Fragmentation Index - Formal Definition}
\begin{itemize}
\item Assume $t_{i}^{M}=1$ for all steps.
\item Assume independent step success $q_{i}$. A step that does not succeed
is said to fail.
\item Write $F$ for the set of steps that fail.
\item Write $\mathcal{C}=\{C_{1},...,C_{k}\}$ for the random variable representing
the collection of maximal connected components of non-failed steps.
\item The weight of each $C_{j}\in\mathcal{C}$ is defined to be minimum
of its augmentation cost or sum of it manual steps
\[
w(C_{j})=min\{1,\;\sum_{i\in C_{j}}t_{i}^{H}\}.
\]
\item Given a realization of $C$ and $F$, the realized fragmentation is:
\[
\sum_{i\in F}\min\,\{t_{i}^{H},\frac{t_{i}^{M}}{q_{i}}\}+\sum_{C_{j}\in\mathcal{C}}w(C_{j}).
\]
\item Fragmentation index is the expected value of the realized fragmentation
above. \hyperlink{Fragmentation_Main}{\beamergotobutton{Return}}
\end{itemize}
\end{frame}
%
\label{A3:Aggregation}
\begin{frame}[noframenumbering]{Micro to Macro Production Function Aggregation}
\begin{itemize}
\item Production function aggregation setup:
\begin{itemize}
\item Firms use (1) skill-adjusted AI management labor, (2) skill-adjusted
human labor, (3) capital.
\item Assume capital is plenty and is a fixed input so firms decide on (1)
and (2) conditional on capital stock.
\item Different types of labor have different base wage rates: $w_{M}$
for AI management and $w_{H}$ for human labor.
\item Updated job wage equation:
\[
\text{Wage}_{j}=w_{H}\left(\sum_{T_{b}^{H}\in J}c_{b}^{H}\right)+w_{M}\left(\sum_{T_{b}^{M}\in J}c_{b}^{M}\right).
\]
\item Skill-adjusted time requirement of task $b$ is:
\[
\tau_{b}=\frac{w_{H}\left(\sum_{T_{\ell}^{H}\in J(b)}c_{\ell}^{H}\right)+w_{M}\left(\sum_{T_{\ell}^{M}\in J(b)}c_{\ell}^{M}\right)}{w_{O(b)}}\,t_{b},
\]
where $O(b)=H$ if $b$ is a manual task and $O(b)=M$ if $b$ is
an AI chain.
\item Wage bill paid to task $b$ is thus $w_{H}\tau_{b}$ if executed manually
and $w_{M}\tau_{b}\alpha^{-d_{b}}$ if executed via AI.
\end{itemize}
\end{itemize}
\end{frame}
%
\begin{frame}[noframenumbering]{(1) Within-Firm Aggregation}
\begin{itemize}
\item Fix automation strategy $\mathcal{T}$ and job design $\mathcal{J}$.
\item Assume hand-offs are performed manually $\rightarrow$ model as an
additional $p-1$ manual tasks where $p$ is number of jobs.
\item With hand-offs being standalone tasks $\rightarrow$ reorder tasks
in the production sequence: $1,...,k$ are AI chains, and $k+1,...n+p-1$
are manual tasks.
\item Individual firm's production function is Leontief in steps:
\[
x=\min\left\{ \frac{l_{1}}{\tau_{1}^{M}\,\alpha^{-d_{1}}},\,\cdots,\,\frac{l_{k}}{\tau_{k}^{M}\,\alpha^{-d_{k}}},\,\frac{l_{k+1}}{\tau_{k+1}^{H}},\,\cdots,\,\frac{l_{n}}{\tau_{n}^{H}},\,\frac{l_{n+1}}{\tau^{S}(J_{1})},\,\cdots,\,\frac{l_{n+p-1}}{\tau^{S}(J_{p-1})}\right\} ,
\]
where $x$ is output and $l_{i}$ is labor assigned to task $i$.
\end{itemize}
\end{frame}
%
\begin{frame}[noframenumbering]{(1) Within-Firm Aggregation - Continued}

\[
x=\min\left\{ \frac{l_{1}}{\tau_{1}^{M}\,\alpha^{-d_{1}}},\,\cdots,\,\frac{l_{k}}{\tau_{k}^{M}\,\alpha^{-d_{k}}},\,\frac{l_{k+1}}{\tau_{k+1}^{H}},\,\cdots,\,\frac{l_{n}}{\tau_{n}^{H}},\,\frac{l_{n+1}}{\tau^{S}(J_{1})},\,\cdots,\,\frac{l_{n+p-1}}{\tau^{S}(J_{p-1})}\right\} .
\]

\begin{itemize}
\item Can be represented as:
\[
x=\min\,\left\{ \frac{\bar{\alpha}\,l_{M}}{\tau_{M}},\,\frac{l_{H}}{\tau_{H}}\right\} ,
\]
where $\bar{\alpha}$ is the effective AI quality level and:
\[
\tau_{M}=\sum_{b=1}^{k}\tau_{b}^{M},\qquad\qquad\tau_{H}=\sum_{j=1}^{p-1}\tau^{S}(J_{j})+\sum_{b=k+1}^{n}\tau_{b}^{H},\qquad\qquad\bar{\alpha}=\frac{\sum_{b=1}^{k}\tau_{b}^{M}}{\sum_{b=1}^{k}\tau_{b}^{M}\alpha^{-d_{b}}},
\]
if the rate of substitution of inputs within groups are independent
of rate of substitution in the other group (which is satisfied in
the Leontief case).
\item This representation gives insight into how much AI management versus
human labor is used at the firm level.
\end{itemize}
\end{frame}
%
\begin{frame}[noframenumbering]{(2) Across-Firm Aggregation}
\begin{itemize}
\item Consider a unit mass of firms. 
\item Firms do not know their actual AI quality parameter before production
occurs; they make decisions based on a common prior.
\item Production unfolds in two stages:
\begin{itemize}
\item Stage 1: firms decide on organization structure (i.e., automation
strategy $\mathcal{T}$ and job design $\mathcal{J}$) based on expected
effective AI quality level.
\item Stage 2: effective AI quality is realized and firms produce.
\end{itemize}
\item We show that with proper heterogeneity in effective AI quality levels
micro production functions can aggregate into a CES production function
of the form: 
\[
X=\left(\theta_{M}\,M^{\rho}+\theta_{H}\,H^{\rho}+(1-\theta_{M}-\theta_{H})\,K^{\rho}\right)^{\frac{1}{\rho}},
\]
where capital can be normalized to 1 without loss.
\end{itemize}
\end{frame}
%
\begin{frame}[noframenumbering]{(2) Across-Firm Aggregation - Continued}
\begin{itemize}
\item Normalize the output price $p=1$, so that $w_{M}$ and $w_{H}$ are
real wage rates. A firm's profitability condition implies:
\[
w_{M}\,l_{M}+w_{H}\,l_{H}\leq x.
\]
\item In equilibrium:
\[
x=\frac{\bar{\alpha}\,l_{M}}{\tau_{M}}=\frac{l_{H}}{\tau_{H}}.
\]
Substituting for $l_{H}$ and $x$ in terms of $\tau_{M}$ and $\tau_{H}$
we obtain:
\[
w_{M}\,l_{M}+w_{H}\,\frac{\tau_{H}\,\bar{\alpha}\,l_{M}}{\tau_{M}}\leq\frac{\bar{\alpha}\,l_{M}}{\tau_{M}}.
\]
\item Rearrange terms to get:
\[
\gamma=\frac{w_{M}\,\tau_{M}}{1-w_{H}\,\tau_{H}}\leq\bar{\alpha}\leq1.
\]
\item Let $\phi(\bar{\alpha})$ be the dist. of output in terms of effective
AI quality level of firms. By definition:
\[
X=\int_{\gamma}^{1}\,\phi(\bar{\alpha})\,d\bar{\alpha},\qquad M=\int_{\gamma}^{1}\,\tau_{M}\frac{\phi(\bar{\alpha})}{\bar{\alpha}}\,d\bar{\alpha},\qquad H=\int_{\gamma}^{1}\,\tau_{H}\,\phi(\bar{\alpha})\,d\bar{\alpha}.
\]
\end{itemize}
\end{frame}
%
\begin{frame}[noframenumbering]{(2) Across-Firm Aggregation - Continued}
\begin{itemize}
\item We have $X,M,H$, all in terms of $\phi(\bar{\alpha})$ in
\[
X=\left(\theta_{M}\,M^{\rho}+\theta_{H}\,H^{\rho}+(1-\theta_{M}-\theta_{H})\right)^{\frac{1}{\rho}}.
\]
\item Solving for $\phi(\bar{\alpha})$ when $\rho<0$ (i.e., $\sigma<1$)
yields the following closed form:{\tiny
\[
\phi(\bar{\alpha})=\frac{\left(1-\theta_{M}-\theta_{H}\right)^{\frac{1}{\rho}}\left(1-\theta_{H}\,\tau_{H}^{\rho}\right)^{\frac{\rho}{\rho-1}}\left(\theta_{M}\,\tau_{M}^{\rho}\right)^{\frac{1}{1-\rho}}}{1-\rho}(\bar{\alpha})^{\frac{1}{\rho-1}}\left[1-\theta_{H}\,\tau_{H}^{\rho}-\left(1-\theta_{H}\,\tau_{H}^{\rho}\right)^{\frac{\rho}{\rho-1}}\left(\theta_{M}\,\tau_{M}^{\rho}\right)^{\frac{1}{1-\rho}}(\bar{\alpha})^{\frac{\rho}{\rho-1}}\right]^{-\frac{1+\rho}{\rho}}.
\]
}{\tiny\par}
\end{itemize}
\begin{itemize}
\item \textit{...Conjecture...}: For certain parameter values $(\theta_{M},\theta_{H},\rho)$
a first-order stochastic shift in effective AI quality level $\bar{\alpha}$
will increase the ratio of $\tfrac{M}{H}$, implying that (relatively)
more AI management labor will be employed in the aggregate economy.
\item \textit{...Caveat...}: We've fixed automation strategy $\mathcal{T}$
and job design $\mathcal{J}$; bear in mind that production function
frontier is non-smooth.
\end{itemize}
\hyperlink{Aggregation_Main}{\beamergotobutton{Return}}
\end{frame}
%
\label{A4:GPT_TaskOrderingPrompt}
\begin{frame}{GPT-5-mini Prompt for Task Ordering}

\begin{figure}[htbp]
    \begin{center}
    \includegraphics[width=0.75\linewidth]{plots/gpt_prompt1.png}
    \end{center}
\end{figure}\hyperlink{Data_Sources}{\beamergotobutton{Return}}
\end{frame}
%
\label{A4:GPT_DWAsimilarTasksPrompt}
\begin{frame}{GPT-5-mini Prompt for Finding Similar Tasks in a DWA}

\begin{figure}[htbp]
    \begin{center}
    \includegraphics[width=0.55\linewidth]{plots/gpt_prompt2.png}
    \end{center}
\end{figure}\hyperlink{DWA}{\beamergotobutton{Return}}
\end{frame}

\end{document}
